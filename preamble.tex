
%%%%% deal with the excessive number of packages %%%%%
\usepackage{etex}

\usepackage{silence} % suppres font shape warnings
  \WarningFilter{latexfont}{Font shape}
\usepackage{ifthen} % for conditional macros

%%%%% general math %%%%%
\usepackage[tbtags]{mathtools} % loads amsmath
\usepackage{amssymb,stmaryrd}
  % \SetSymbolFont{stmry}{bold}{U}{stmry}{m}{n}
\usepackage{relsize} % change size of math operators
\usepackage{exscale} % change size of math operators arbitrarily
\usepackage{scalerel} % change size of math delimiters
% prevent align env at top of minipage from adding additional padding
\usepackage{etoolbox}
  \makeatletter
  \pretocmd\start@align{%
    \if@minipage\kern-\topskip\kern-\abovedisplayskip\fi
  }{}{}
  \makeatother

%%% all-purpose math macros
% parens
\DeclarePairedDelimiterX\PARENS[1](){#1}
\newcommand{\p}[1]{\PARENS*{#1}}
% set comprehension: \set{ ... \giv ... } = { ... | ... }
\providecommand{\giv}{}
\DeclarePairedDelimiterX\SET[1]\{\}{%
  \renewcommand{\giv}{\nonscript\:\delimsize\vert\nonscript\:\mathopen{}}
  #1
}
\newcommand{\set}[1]{\SET*{#1}}
% grand union: \uset{ ... \giv ... } = U{ ... | ... }
\DeclarePairedDelimiterXPP\USET[1]\bigcup\{\}{}{%
  \renewcommand{\giv}{\nonscript\:\delimsize\vert\nonscript\:\mathopen{}}
  #1
}
\newcommand{\uset}[1]{\USET*{#1}}
% pairs: \pair{..}{..} = <.., ..>
\DeclarePairedDelimiterX\PAIR[1]\langle\rangle{#1}
\newcommand{\pair}[2]{\PAIR*{#1,\pt #2}}
\newcommand{\thickpair}[2]{\PAIR*{#1,\ \ #2}}
\newcommand{\typepair}[2]{#1 \ast #2}

%%%%% examples, lists, footnotes, citations %%%%%
\usepackage[bottom,multiple]{footmisc} % force footnotes to bottom of page
\usepackage{enumitem} % customize lists
\usepackage{epltxfn} % expex examples in footnotes
\usepackage{expex} % for example sentences
\newcommand{\excite}[3][]{% citation at right edge of example
  \ifthenelse{\equal{#1}{}}{% no opt arg (no prefix) 
    \ifthenelse{\equal{#3}{ibid}}{% no reference to look up
      \rightcomment{[\textit{ibid}:~#2]}
    }{% look up reference
      \rightcomment{[\citealt[#2]{#3}]}
    }
  }{% optional argument present
    \ifthenelse{\equal{#1}{citealias}}{% use citealias
      \rightcomment{[\citetalias{#3}:~#2]}
    }{% set opt arg as citation prefix
      \ifthenelse{\equal{#3}{ibid}}{% no reference to look up
        \rightcomment{[#1 \textit{ibid}:~#2]}
      }{% look up reference
        \rightcomment{[#1 \citealt[#2]{#3}]}
      }
    }
  }
}
% inline citation style abbreviations
\usepackage{natbib}
  \bibpunct[: ]{(}{)}{,}{a}{}{,}
  \newcommand{\BIBand}{\&}
  \setlength{\bibsep}{0pt}
  \setlength{\bibhang}{0.25in}
  \bibliographystyle{sp}
  \newcommand{\posscitet}[1]{\citeauthor{#1}'s (\citeyear{#1})}
  \newcommand{\possciteauthor}[1]{\citeauthor{#1}'s}
  \newcommand{\pgposscitet}[2]{\citeauthor{#1}'s (\citeyear{#1}:~#2)}
  \newcommand{\secposscitet}[2]{\citeauthor{#1}'s (\citeyear{#1}:~$\S$#2)}
  \newcommand{\pgcitealt}[2]{\citealt{#1}:~#2}
  \newcommand{\seccitealt}[2]{\citealt{#1}:~$\S$#2}
  \newcommand{\pgcitep}[2]{(\citealt{#1}:~#2)}
  \newcommand{\seccitep}[2]{(\citealt{#1}:~$\S$#2)}
  \newcommand{\pgcitet}[2]{\citeauthor{#1} (\citeyear{#1}:~#2)}
  \newcommand{\seccitet}[2]{\citeauthor{#1} (\citeyear{#1}:~$\S$#2)}



%%%%% tables %%%%%
\usepackage{array} % more control over column styles
\usepackage{booktabs} % better rules

%%% array columns without spaces
\newcolumntype{L}{@{}l@{}}
\newcolumntype{R}{@{}r@{}}
\newcolumntype{C}{@{}c@{}}



%%%%% layout and formatting %%%%%
\usepackage[margin=1in]{geometry} % margins
\usepackage{sectsty} % customize section title appearance
  \allsectionsfont{\normalsize}

%%% customize the abstract
\newcommand{\frontmatterspacing}[1]{%
  \small
  \topsep 10pt
  \advance\topsep by 3.5ex plus -1ex minus -0.2ex
  \setlength{\listparindent}{0em}
  \setlength{\itemindent}{0em}
  \setlength{\leftmargin}{#1}
  \setlength{\rightmargin}{\leftmargin}
  \setlength{\parskip}{0em}
}
\renewenvironment{abstract}{%
  \list{}{\frontmatterspacing{0.25in}}%
  \item\relax\textbf{\abstractname}
}{\endlist}

%%% font preferences
\usepackage[T1]{fontenc}
\usepackage[utf8]{inputenc}
\usepackage{libertine}
\usepackage[libertine,liby,bigdelims]{newtxmath}

%%% links
\RequirePackage[usenames]{xcolor}
\definecolor{splinkcolor}{rgb}{.0,.2,.4}
\RequirePackage[colorlinks,breaklinks,
                linkcolor=splinkcolor, 
                urlcolor=splinkcolor, 
                citecolor=splinkcolor,
                filecolor=splinkcolor,
                plainpages=false,
                pdfpagelabels,
                bookmarks=false,
                pdfstartview=FitH]{hyperref}
\newcommand{\doi}[1]{\url{http://dx.doi.org/#1}}
\urlstyle{rm}
\usepackage{hyperref}


%%%%% additional symbols and macros %%%%%%

%%% sp macros
\newcommand{\sv}[1]{\llbracket #1 \rrbracket}
\newcommand{\with}{\mathbin{\&}}

%%% basic convenience macros
\newcommand{\pt}{\hspace{1pt}}
\newcommand{\ppt}{\hspace{2pt}}
\newcommand{\col}{\pt{:}\ppt}
\newcommand{\dt}{\pt{.}\ppt}
\newcommand{\cn}[1]{{\sf #1}}
\newcommand{\cat}{\hspace{-2pt}\cdot\hspace{-2pt}}
\newcommand{\sto}{\mathbin{\shortrightarrow}}
\newcommand{\must}{\mathlarger{\mathlarger{\Box}}}
\newcommand{\rest}[2]{#1 \col #2}

\newcommand*\phantomas[3][c]{% phantom to width of second arg
   \ifmmode
     \makebox[\widthof{$#2$}][#1]{$#3$}%
   \else
     \makebox[\widthof{#2}][#1]{#3}%
   \fi
}
\newcommand{\objl}[1]{`#1'}

%%% judgment diacritics
\newcommand{\att}{${}^{\boldsymbol\gamma}$}
\newcommand{\yes}{\textsuperscript{\checkmark}}
\newcommand{\bad}{\textsuperscript{\small\#}}
\newcommand{\mar}{\textsuperscript{?}}
\newcommand{\ung}{*}

%%% monads and towers
\renewcommand{\l}{\lambda}
\newcommand{\mtype}[2]{\mathbb{#1}_{#2}}
\newcommand{\bind}{\star}
\newcommand{\unit}{\eta}
\newcommand{\nil}{\varepsilon}
\newcommand{\hole}{[\,]}
\newcommand{\fsl}{\,/\,}
\newcommand{\bsl}{\,\backslash\,}
\newcommand{\FSL}[1]{\stretchrel[600]{/}{#1}}
\newcommand{\BSL}[1]{\stretchrel[600]{\backslash}{#1}}

% bipartite 2-story tower
\newcommand{\bitt}[2]{%
  \ensuremath{%
    \mathinner{%
      \begin{tabular}[c]{@{}c@{}}
        {\ensuremath{#1}}\\
        \hline
        {\ensuremath{#2}}\\
      \end{tabular}
    }
  }
}
% bipartite 3-story tower
\newcommand{\bittt}[3]{%
  \ensuremath{%
    \mathinner{%
      \begin{tabular}[c]{@{}c@{}}
        {\ensuremath{#1}}\\
        \hline
        {\ensuremath{#2}}\\
        \hline
        {\ensuremath{#3}}\\
      \end{tabular}
    }
  }
}
% bipartite 4-story tower
\newcommand{\bitttt}[4]{%
  \ensuremath{%
    \mathinner{%
      \begin{tabular}[c]{@{}c@{}}
        {\ensuremath{#1}}\\
        \hline
        {\ensuremath{#2}}\\
        \hline
        {\ensuremath{#3}}\\
        \hline
        {\ensuremath{#4}}\\
      \end{tabular}
    }
  }
}
% tripartite 2-story tower
\newcommand\tritt[3]{%
  \ensuremath{%
    \mathinner{
      \begin{tabular}[c]{@{ }c@{ }}
        \strut\ensuremath{#1}\ \hfil\vrule\hfil\ \ensuremath{#2}\\
        \hline
        \strut\ensuremath{#3}\\
      \end{tabular}
    }
  }
}


%%%%% drawing %%%%%
\usepackage{tikz-qtree}

%%% tikz (for matrices)
\usepackage{tikz}
\usetikzlibrary{calc}


%%%%% borrowed shapes from other math libraries %%%%%

%%% corners and bigplus from mathabx w/o loading the whole package
\DeclareFontFamily{U}{mathb}{}
\DeclareFontShape{U}{mathb}{m}{n}{%
      <5> <6> <7> <8> <9> <10> gen * mathb
      <10.95> mathb10 <12> <14.4> <17.28> <20.74> <24.88> mathb12 }{}
\DeclareSymbolFont{mathb}{U}{mathb}{m}{n}
\DeclareMathSymbol{\lcorners}{4}{mathb}{"76}% name to be checked
\DeclareMathSymbol{\rcorners}{5}{mathb}{"77}% name to be checked
\DeclareMathSymbol{\acorner}{4}{mathb}{"78}% name to be checked
\DeclareMathSymbol{\bcorner}{5}{mathb}{"79}% name to be checked
\DeclareMathSymbol{\ccorner}{4}{mathb}{"7A}% name to be checked
\DeclareMathSymbol{\dcorner}{5}{mathb}{"7B}% name to be checked
\DeclareFontFamily{U}{mathx}{\hyphenchar\font45}
\DeclareFontShape{U}{mathx}{m}{n}{%
    <5> <6> <7> <8> <9> <10>
    <10.95> <12> <14.4> <17.28> <20.74> <24.88>
    mathx10
}{}
\DeclareSymbolFont{mathx}{U}{mathx}{m}{n}
\DeclareMathSymbol{\bigplus}{1}{mathx}{"90}

%%% powerset symbol from mnsymbol
\DeclareFontFamily{U}{MnSymbolC}{}
\DeclareSymbolFont{mnsymbols}{U}{MnSymbolC}{m}{n}
\DeclareFontShape{U}{MnSymbolC}{m}{n}{%
<-6> MnSymbolC5
<6-7> MnSymbolC6
<7-8> MnSymbolC7
<8-9> MnSymbolC8
<9-10> MnSymbolC9
<10-12> MnSymbolC10
<12-> MnSymbolC12}{}
\DeclareMathSymbol{\powerset}{\mathop}{mnsymbols}{180}

%%% right triangles from wasysym
\DeclareFontFamily{U}{wasy}{}
\DeclareFontShape{U}{wasy}{m}{n}{%
<-6> wasy5
<6-7> wasy6
<7-8> wasy7
<8-9> wasy8
<9-10> wasy9
<10-> wasy10}{}
\DeclareSymbolFont{wasy}{U}{wasy}{m}{n}
\DeclareMathSymbol{\RHD}{\mathbin}{wasy}{"11}
\let\rhd\undefined
\DeclareMathSymbol\rhd{\mathbin}{wasy}{"03}
