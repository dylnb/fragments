\documentclass[10pt,fleqn]{article}


%%%%% deal with the excessive number of packages %%%%%
\usepackage{etex}

\usepackage{silence} % suppres font shape warnings
  \WarningFilter{latexfont}{Font shape}
\usepackage{ifthen} % for conditional macros

%%%%% general math %%%%%
\usepackage[tbtags]{mathtools} % loads amsmath
\usepackage{amssymb,stmaryrd}
  % \SetSymbolFont{stmry}{bold}{U}{stmry}{m}{n}
\usepackage{relsize} % change size of math operators
\usepackage{exscale} % change size of math operators arbitrarily
\usepackage{scalerel} % change size of math delimiters
% prevent align env at top of minipage from adding additional padding
\usepackage{etoolbox}
  \makeatletter
  \pretocmd\start@align{%
    \if@minipage\kern-\topskip\kern-\abovedisplayskip\fi
  }{}{}
  \makeatother

%%% all-purpose math macros
% parens
\DeclarePairedDelimiterX\PARENS[1](){#1}
\newcommand{\p}[1]{\PARENS*{#1}}
% set comprehension: \set{ ... \giv ... } = { ... | ... }
\providecommand{\giv}{}
\DeclarePairedDelimiterX\SET[1]\{\}{%
  \renewcommand{\giv}{\nonscript\:\delimsize\vert\nonscript\:\mathopen{}}
  #1
}
\newcommand{\set}[1]{\SET*{#1}}
% grand union: \uset{ ... \giv ... } = U{ ... | ... }
\DeclarePairedDelimiterXPP\USET[1]{\!\bigcup}\{\}{}{%
  \renewcommand{\giv}{\nonscript\:\delimsize\vert\nonscript\:\mathopen{}}
  #1
}
\newcommand{\uset}[1]{\USET*{#1}}
% tuples
\DeclarePairedDelimiterX\TUP[1]\langle\rangle{#1}
% tuple of arbitrary length
\makeatletter
\newcommand{\tup}[1]{%
  \ensuremath{%
    \TUP*{\my@tups #1,\relax\noexpand\@eolst}%
  }%
}
\def\my@tups #1,#2\@eolst{%
  \ifx\relax#2\relax
    #1%
  \else
    #1,\pt \my@tups #2\@eolst%
  \fi
}
\makeatother
\newcommand{\pair}[2]{\tup{#1, #2}}
\newcommand{\thickpair}[2]{\TUP*{#1,\ \ #2}}
\newcommand{\typepair}[2]{#1 \ast #2}
% verts
\DeclarePairedDelimiterX\ABS[1]||{#1}
\newcommand{\abs}[1]{\ABS*{#1}}

%%%%% examples, lists, footnotes, citations %%%%%
\usepackage[bottom,multiple]{footmisc} % force footnotes to bottom of page
\usepackage{enumitem} % customize lists
\usepackage{epltxfn} % expex examples in footnotes
\usepackage{expex} % for example sentences
\newcommand{\excite}[3][]{% citation at right edge of example
  \ifthenelse{\equal{#1}{}}{% no opt arg (no prefix) 
    \ifthenelse{\equal{#3}{ibid}}{% no reference to look up
      \rightcomment{[\textit{ibid}:~#2]}
    }{% look up reference
      \rightcomment{[\citealt[#2]{#3}]}
    }
  }{% optional argument present
    \ifthenelse{\equal{#1}{citealias}}{% use citealias
      \rightcomment{[\citetalias{#3}:~#2]}
    }{% set opt arg as citation prefix
      \ifthenelse{\equal{#3}{ibid}}{% no reference to look up
        \rightcomment{[#1 \textit{ibid}:~#2]}
      }{% look up reference
        \rightcomment{[#1 \citealt[#2]{#3}]}
      }
    }
  }
}
% inline citation style abbreviations
\usepackage{natbib}
  \bibpunct[: ]{(}{)}{,}{a}{}{,}
  \newcommand{\BIBand}{\&}
  \setlength{\bibsep}{0pt}
  \setlength{\bibhang}{0.25in}
  \bibliographystyle{sp}
  \newcommand{\posscitet}[1]{\citeauthor{#1}'s (\citeyear{#1})}
  \newcommand{\possciteauthor}[1]{\citeauthor{#1}'s}
  \newcommand{\pgposscitet}[2]{\citeauthor{#1}'s (\citeyear{#1}:~#2)}
  \newcommand{\secposscitet}[2]{\citeauthor{#1}'s (\citeyear{#1}:~$\S$#2)}
  \newcommand{\pgcitealt}[2]{\citealt{#1}:~#2}
  \newcommand{\seccitealt}[2]{\citealt{#1}:~$\S$#2}
  \newcommand{\pgcitep}[2]{(\citealt{#1}:~#2)}
  \newcommand{\seccitep}[2]{(\citealt{#1}:~$\S$#2)}
  \newcommand{\pgcitet}[2]{\citeauthor{#1} (\citeyear{#1}:~#2)}
  \newcommand{\seccitet}[2]{\citeauthor{#1} (\citeyear{#1}:~$\S$#2)}



%%%%% tables %%%%%
\usepackage{array} % more control over column styles
\usepackage{booktabs} % better rules

%%% array columns without spaces
\newcolumntype{L}{@{}l@{}}
\newcolumntype{R}{@{}r@{}}
\newcolumntype{C}{@{}c@{}}



%%%%% layout and formatting %%%%%
\usepackage[margin=1in]{geometry} % margins
\usepackage{sectsty} % customize section title appearance
  \allsectionsfont{\normalsize}

%%% customize the abstract
\newcommand{\frontmatterspacing}[1]{%
  \small
  \topsep 10pt
  \advance\topsep by 3.5ex plus -1ex minus -0.2ex
  \setlength{\listparindent}{0em}
  \setlength{\itemindent}{0em}
  \setlength{\leftmargin}{#1}
  \setlength{\rightmargin}{\leftmargin}
  \setlength{\parskip}{0em}
}
\renewenvironment{abstract}{%
  \list{}{\frontmatterspacing{0.25in}}%
  \item\relax\textbf{\abstractname}
}{\endlist}

%%% font preferences
\usepackage[T1]{fontenc}
\usepackage[utf8]{inputenc}
\usepackage{libertine}
\usepackage[libertine,liby,bigdelims]{newtxmath}

%%% links
\RequirePackage[usenames]{xcolor}
\definecolor{splinkcolor}{rgb}{.0,.2,.4}
\RequirePackage[colorlinks,breaklinks,
                linkcolor=splinkcolor, 
                urlcolor=splinkcolor, 
                citecolor=splinkcolor,
                filecolor=splinkcolor,
                plainpages=false,
                pdfpagelabels,
                bookmarks=false,
                pdfstartview=FitH]{hyperref}
\newcommand{\doi}[1]{\url{http://dx.doi.org/#1}}
\urlstyle{rm}
\usepackage{hyperref}


%%%%% additional symbols and macros %%%%%%

%%% sp macros
\newcommand{\sv}[1]{\llbracket #1 \rrbracket}
\newcommand{\with}{\mathbin{\&}}

%%% basic convenience macros
\newcommand{\pt}{\hspace{1pt}}
\newcommand{\ppt}{\hspace{2pt}}
\newcommand{\col}{\pt{:}\ppt}
\newcommand{\dt}{\pt{.}\ppt}
\newcommand{\cn}[1]{{\sf #1}}
\newcommand{\cat}{\hspace{-2pt}\cdot\hspace{-2pt}}
\newcommand{\sto}{\mathbin{\shortrightarrow}}
\newcommand{\must}{\mathlarger{\mathlarger{\Box}}}
\newcommand{\rest}[2]{#1 \col #2}
\newcommand{\objl}[1]{`#1'}
\newcommand{\tru}{\textsf{\bfseries T}}
\newcommand{\fals}{\textsf{\bfseries F}}
\newcommand{\ms}[1]{\mathsmaller{#1}}

% phantom to width of second argument
\newcommand*\phantomas[3][c]{%
   \ifmmode
     \makebox[\widthof{$#2$}][#1]{$#3$}%
   \else
     \makebox[\widthof{#2}][#1]{#3}%
   \fi
}
% dots to break sections in handouts, etc
\newcommand{\dotbreak}[1][]{%
  % \vspace{0.5em}\dotfill\hspace{0.5\textwidth} \\ \textsc{#1}\vspace{0.5em}
  \ifthenelse{\equal{#1}{}}{%
    \vspace{0.5em}\dotfill\vspace{0.5em}
  }{%
    \vspace{0.5em}\dotfill\hspace{0.5\textwidth} \\ \textsc{#1}\vspace{0.5em}
  }
}
% two-column minipage, with widths controlled by optional param
\newenvironment{minisplit}[1][0.5]{%
  \newcommand{\splitmini}{%
    \end{minipage}%
    \begin{minipage}[t]{\dimexpr\textwidth-#1\textwidth\relax}%
  }
  \noindent\ignorespaces%
  \begin{minipage}[t]{#1\textwidth}%
}{%
  \end{minipage}%
  \ignorespacesafterend%
}

%%% judgment diacritics
\newcommand{\att}{${}^{\boldsymbol\gamma}$}
\newcommand{\yes}{\textsuperscript{\checkmark}}
\newcommand{\bad}{\textsuperscript{\small\#}}
\newcommand{\mar}{\textsuperscript{?}}
\newcommand{\ung}{*}

%%% monads and towers
\renewcommand{\l}{\lambda}
\newcommand{\mtype}[2]{\mathbb{#1}_{#2}}
\newcommand{\bind}{\star}
\newcommand{\unit}{\eta}
\newcommand{\nil}{\varepsilon}
\newcommand{\dnar}{\downarrow}
\newcommand{\upar}{\uparrow}
\newcommand{\reset}{\downupharpoons}
\newcommand{\hole}{[\,]}
\newcommand{\fsl}{\sslash}
\newcommand{\bsl}{\bbslash}
\newcommand{\msl}{\,\|\,}
\newcommand{\FSL}[1]{\ \stretchrel[600]{\ \fsl\ }{#1}}
\newcommand{\BSL}[1]{\ \stretchrel[600]{\ \bsl\ }{#1}}
\newcommand{\MSL}[1]{\ \stretchrel[600]{\ \msl\ }{#1}}

\makeatletter
\newcommand{\tower}[2][\my@btows]{%
  \ensuremath{\mathinner{%
      \begin{tabular}[c]{@{}>{$\displaystyle}c<{$}@{}}%
        #1 #2,\relax\noexpand\@eolst%
      \end{tabular}%
  }}%
}
\def\my@ttows #1,#2,#3\@eolst{%
  \ifx\relax#3\relax
    \strut\ensuremath{#1}\\
  \else
    \strut\ensuremath{#1}\ \hfil\vrule\hfil\ \ensuremath{#2}\\
    \hline
    \my@ttows #3\@eolst%
  \fi
}
\def\my@btows #1,#2\@eolst{%
  \ifx\relax#2\relax
    \ensuremath{#1}%
  \else
    \ensuremath{#1}\\\hline%
    \my@btows #2\@eolst%
  \fi
}
% tripartite tower of aribitrary depth
\newcommand{\ttower}[1]{\tower[\my@ttows]{#1}}
% bipartite tower of arbitrary depth
\newcommand{\btower}[1]{\tower{#1}}
\makeatother

\newcommand{\bitt}[2]{\btower{#1, #2}}
\newcommand{\bittt}[3]{\btower{#1, #2, #3}}
\newcommand{\bitttt}[4]{\btower{#1, #2, #3, #4}}

% tripartite 2-story tower
\newcommand{\tritt}[3]{%
  \ensuremath{\mathinner{%
      \begin{tabular}[c]{@{ }c@{ }}
        \strut\ensuremath{#1}\ \hfil\vrule\hfil\ \ensuremath{#2}\\
        \hline
        \strut\ensuremath{#3}\\
      \end{tabular}
  }}
}


%%%%% drawing %%%%%
\usepackage{tikz-qtree}

%%% tikz (for matrices)
\usepackage{tikz}
\usetikzlibrary{calc}


%%%%% borrowed shapes from other math libraries %%%%%

%%% corners, bigplus, harpoons from mathabx w/o loading the whole font
% corners
\DeclareFontFamily{U}{mathb}{}
\DeclareFontShape{U}{mathb}{m}{n}{%
      <5> <6> <7> <8> <9> <10> gen * mathb
      <10.95> mathb10 <12> <14.4> <17.28> <20.74> <24.88> mathb12 }{}
\DeclareSymbolFont{mathb}{U}{mathb}{m}{n}
\DeclareMathSymbol{\lcorners}{4}{mathb}{"76}% name to be checked
\DeclareMathSymbol{\rcorners}{5}{mathb}{"77}% name to be checked
\DeclareMathSymbol{\acorner}{4}{mathb}{"78}% name to be checked
\DeclareMathSymbol{\bcorner}{5}{mathb}{"79}% name to be checked
\DeclareMathSymbol{\ccorner}{4}{mathb}{"7A}% name to be checked
\DeclareMathSymbol{\dcorner}{5}{mathb}{"7B}% name to be checked
% bigplus
\DeclareFontFamily{U}{mathx}{\hyphenchar\font45}
\DeclareFontShape{U}{mathx}{m}{n}{%
    <5> <6> <7> <8> <9> <10>
    <10.95> <12> <14.4> <17.28> <20.74> <24.88>
    mathx10
}{}
\DeclareSymbolFont{mathx}{U}{mathx}{m}{n}
\DeclareMathSymbol{\bigplus}{1}{mathx}{"90}
% harpoons
\DeclareFontFamily{U}{matha}{}
\DeclareFontShape{U}{matha}{m}{n}{%
  <5> <6> <7> <8> <9> <10> gen * matha
  <10.95> matha10 <12> <14.4> <17.28> <20.74> <24.88> matha12 }{}
\DeclareSymbolFont{matha}{U}{matha}{m}{n}
\DeclareMathSymbol{\downupharpoons}{3}{matha}{"EB}

%%% powerset symbol from mnsymbol
\DeclareFontFamily{U}{MnSymbolC}{}
\DeclareSymbolFont{mnsymbols}{U}{MnSymbolC}{m}{n}
\DeclareFontShape{U}{MnSymbolC}{m}{n}{%
<-6> MnSymbolC5
<6-7> MnSymbolC6
<7-8> MnSymbolC7
<8-9> MnSymbolC8
<9-10> MnSymbolC9
<10-12> MnSymbolC10
<12-> MnSymbolC12}{}
\DeclareMathSymbol{\powerset}{\mathop}{mnsymbols}{180}

%%% right triangles from wasysym
\DeclareFontFamily{U}{wasy}{}
\DeclareFontShape{U}{wasy}{m}{n}{%
<-6> wasy5
<6-7> wasy6
<7-8> wasy7
<8-9> wasy8
<9-10> wasy9
<10-> wasy10}{}
\DeclareSymbolFont{wasy}{U}{wasy}{m}{n}
\DeclareMathSymbol{\RHD}{\mathbin}{wasy}{"11}
\let\rhd\undefined
\DeclareMathSymbol\rhd{\mathbin}{wasy}{"03}

\geometry{landscape, margin=0.5in}
\setlength{\parindent}{0pt}
\setlength{\mathindent}{0pt}
% \delimitershortfall=-1pt
\setlist{leftmargin=*}
\newcommand{\ctype}{\mathcal{C}}
\newcommand{\one}{\textbf{1}}
\newcommand{\post}[2]{#1^{#2}}
% \allsectionsfont{\normalsize\normalfont\scshape}

\newcommand{\M}{\text{\sffamily\bfseries M}}
\newcommand{\F}{\text{\sffamily\bfseries F}}
\newcommand{\T}{\text{\sffamily\bfseries T}}

\begin{document}

\section{The Grammar}

\vspace{-1em}\dotbreak[Scope]
%
\begin{align*}
m \fsl n
\coloneq
&\begin{cases}
  m\,n \quad 
  \textbf{if}\ m \dblcolon \alpha\sto\beta,\ n \dblcolon \alpha \\
  \l k \dt m\,\p{\l f \dt n\,\p{\l x \dt k\,\p{f \fsl x}}} \quad
  \textbf{otherwise}
\end{cases}
%
&
m \bsl n
&\coloneq
\begin{cases}
  n\,m \quad
  \textbf{if}\ n \dblcolon \alpha\sto\beta,\ m \dblcolon \alpha \\
  \l k \dt m\,\p{\l x \dt n\,\p{\l f \dt k\,\p{x \bsl f}}} \quad 
  \textbf{otherwise}
\end{cases}
%
&
m \msl n
&\coloneq
\begin{cases}
  \l x \dt m\,x \land n\,x \quad
  \textbf{if}\ n \dblcolon \alpha\sto\beta,\ m \dblcolon \alpha \\
  \l k \dt m\,\p{\l x \dt n\,\p{\l f \dt k\,\p{f \msl x}}} \quad 
  \textbf{otherwise}
\end{cases}
\end{align*}

\dotbreak[Binding]
%
\begin{align*}
  \eta\,x &\coloneq
  \l gh \dt \set{\tup{x,g,h} \giv x \neq \fals}
  %
  &
  m^\bind &\coloneq
  \l kgh \dt \uset{k\,x\,g'\,h' \giv \tup{x,g',h'} \in m\,g\,h}
  %
  &
  m^\dnar &\coloneq
  m\,\eta
  %
  &
  x^\upar &\coloneq
  \p{\eta\,x}^\bind = \l k \dt k\,x
  %
  &
  m^{\rhd u} &\coloneq
  m^\bind\,\p{\l xgh \dt \set{\tup{x,g^{u\mapsto x},h}}}
\end{align*}

\dotbreak[Evaluation]

\begin{minisplit} % Evaluation
\begin{align*}
  m^\reset &\coloneq
  \p{%
    \l gh \dt
    \set{%
      \tup{x,g',h}
    \giv
      \tup{\cdot,\cdot,h'} \in G,\ \ \tup{x,g',\cdot} \in h'\,G
    }
  }^\bind, \quad \textrm{where}\ G = m^\dnar\,g\,\cn{id} \\
  %
  \cn{true}_g\,m &\coloneq
  \exists\tup{\cdot,\cdot,h} \in G \dt
  h\,G \neq \emptyset, \quad
  \textrm{where}\ G = m^\dnar\,g\,\cn{id}
\end{align*}

\vspace{-1.2em}
\parbox{0.95\textwidth}{%
\begin{itemize}
  \item
    Postsups are evaluated at reset boundaries and at closing time. A
    computation is $\cn{true}$ at an input $g$ if it has a successful output
    \emph{after its postsups have applied}. A computation is not fully reset
    until its postsuppositions have each had a whack at the set of outputs, at
    which point the survivors are collected, and the postsups flushed.
\end{itemize}
}
%
\splitmini
%
\begin{itemize} % Evaluation Notes
  \item
    Following Brasoveanu 2012, contexts are split into two components, an
    assignment function, and a ``postsupposition''. Postsups here have the
    recursive type $\ctype \equiv \set{\alpha\ast\gamma\ast\ctype} \sto
    \set{\alpha\ast\gamma\ast\ctype}$, the type of filters on sets of outputs.
  \item
    The dynamic machinery now bears an intriguing resemblance to something
    Wadler (1994) calls the InputOutput Monad. Assignments play the role of
    input, postsups that of outputs, in the sense that they are simply
    accumulated (composed) over the course of the computation, and used to
    post-process the final result.
\end{itemize}
\end{minisplit}

\dotbreak\vspace{-1em}

\section{Basic Definite Descriptions}

\begin{minisplit} % Basic Definites
\begin{spreadlines}{0pt} % Definite Fragment
\begin{align*}
  \textbf{circle} &\coloneq
  \cn{circ} \\
  %
  \textbf{square} &\coloneq
  \cn{sq} \\
  %
  \textbf{in} &\coloneq
  \cn{in} \\
  %
  \textbf{the}_u &\coloneq
    \l ckgh \dt
    \uset{%
      k\,x\,g'\,h'
    \giv
    \tup{\cn{T},g',h'} \in c\,x\,g^{u\mapsto x}\,\post{h}{\one_u}
    } \\
  %
  \one_u &\equiv 
  \l G \dt
  \begin{cases}
    G &\textbf{if}\ \ \abs{\set{g\,u \mid \tup{\cdot,g,\cdot} \in G}} = 1 \\
    \emptyset &\textbf{otherwise}
  \end{cases}
\end{align*}  
\end{spreadlines}
%
\splitmini
%
\begin{itemize} % Definite Notes
  \item
    $\textbf{the}_u$ is just $\textbf{a}_u$ plus a uniqueness postsup, which
    restruicts what the \emph{global set of outputs} is allowed to look like:
    they must all agree on the value of $u$.
  \item
    Thus the uniqueness effect associated with $\textbf{the}$ is effectively
    \emph{delayed} until some program containing it is evaluated, similar to
    the way that other cardinality impositions have been argued to operate
    (Brasoveanu 2012, Henderson 2014).
\end{itemize}
\end{minisplit}

\dotbreak[Reset]

\begin{minisplit} % Resets
\begin{align*} % Reset Deriv
  \sv{\textrm{the square}}
  %
  &\leadsto
  \p{%
    \btower{%
      \textbf{the}_v\,\p{\l y \dt \hole},
      y
    }
    %
    \BSL{%
    %
    \btower{%
      \hole,
      \textbf{square}
    }}
  }^{\dnar\,\reset}
  %
  \leadsto
  \p{%
    \textbf{the}_v\,\p{\l ygh \dt \set{\tup{\cn{sq}\,y, g, h}}}
  }^\reset \\
  %
  %
  &\leadsto
  \p{%
    \btower{%
      \l gh \dt \uset{\hole\,g^{v\mapsto y}\,\post{h}{\one_v} \giv \cn{sq}\,y},
      y
    }
  }^\reset
  %
  \leadsto
  \btower{%
    \l gh \col \abs{G_v} = 1 \dt
    \uset{\hole\,g^{v\mapsto y}\,h \giv \cn{sq}\,y},
    y
  }
\end{align*}
%
\splitmini
%
\begin{itemize} % Reset Notes
  \item
    Note that resetting $\sv{\textrm{the square}}$ in the last reduction
    step here has no effect on its semantic shape, because it's essentially
    $\sv{\textrm{a square}}$.
  \item
    But it does fix its postsup (which I've switched to representing as a
    \emph{pre}sup, since it now just constrains the input). For any incoming
    $g$, the $G$ of $\one_u^\ms{G}$ will be equal to $\set{\tup{y, g^{v\mapsto
    y}, \one_u} \giv \cn{sq}\,y}$.  Nothing from this set will survive the
    $\one_u$ filter unless all the assignment funcs in $G$ happen to map $v$
    to the same square. That is, unless there's exactly one square available
    to assign $v$ to in the first place.
\end{itemize} 
\end{minisplit}

\dotbreak[Absolute Definite Reading]

\begin{minisplit} % Absolute Definites
\begin{align*} % Absolute Definite Deriv
  &
  \sv{\textrm{the circle in the square}} \\
  %
  %
  &
  \p{%
    \btower{\textbf{the}_u\,\p{\l x \dt \hole}, x}
    %
    \BSL{%
    %
    \p{%
      \btower{\hole, \textbf{circle}}
      %
      \MSL{%
      %
      \btower{\hole, \textbf{in}}
      %
      \FSL{%
      %
      \p{%
        \btower{\textbf{the}_v\,\p{\l y \dt \hole}, y}
        %
        \BSL{%
        %
        \btower{\hole, \textbf{square}}
        }
      }^{\dnar\,\reset} } }
    } }
  }^{\dnar\,\reset} \\
  %
  %
  &
  \p{%
    \btower{\textbf{the}_u\,\p{\l x \dt \hole}, x}
    %
    \BSL{%
    %
    \p{%
      \btower{\hole, \textbf{circle}}
      %
      \MSL{%
      %
      \btower{\hole, \textbf{in}}
      %
      \FSL{%
      %
      \btower{%
        \l gh \col \abs{G_v} = 1 \dt
        \uset{\hole\,g^{v\mapsto y}\,h \giv \cn{sq}\,y},
        y
      } } }
    } }
  }^{\dnar\,\reset} \\
  %
  %
  &
  \p{%
    \btower{%
      \textbf{the}_u\,\p{%
        \l xgh \col \abs{G_v} = 1 \dt
        \uset{\hole\,g^{v\mapsto y}\,h \giv \cn{sq}\,y}
      },
      \cn{circ}\,x \land \cn{in}\,y\,x
    }
  }^{\dnar\,\reset} \\
  %
  %
  &
  \p{%
    \btower{%
      \l gh \col \abs{G_v} = 1 \dt
      \uset{%
        \hole\,g^{\substack{u\mapsto x\\ v\mapsto y}}\,\post{h}{\one_u}
      \giv
        \cn{sq}\,y,\ \cn{circ}\,x,\ \cn{in}\,y\,x
      },
      x
    }
  }^\reset \\
  %
  %
  &
  \btower{%
    \l gh \col \abs{G_v} = \abs{G'_u} = 1 \dt
    \uset{%
      \hole\,g^{\substack{u\mapsto x\\ v\mapsto y}}\,h
    \giv
      \cn{sq}\,y,\ \cn{circ}\,x,\ \cn{in}\,y\,x
    },
    x
  }
\end{align*}
%
\splitmini
%
\begin{itemize} % Absolute Definite Notes
  \item
    The inner definite is reset, freezing its presupposition as above. When
    the input assignment $g$ is eventually inserted, we will have $G =
    \set{\tup{y, g^{v\mapsto y}, \one_v} \giv \cn{sq}\,y}$, and the
    presupposition will guarantee that $g'\,v$ is constant across the outputs.
  \item
    As with the inner DP, the host DP's presupposition is fixed when it is
    reset. This time, we have $G' = \set{\tup{x, g^{\substack{u\mapsto x\\
    v\mapsto y}}, \one_u} \giv \cn{sq}\,y,\ \cn{circ}\,x,\ \cn{in}\,y\,x}$,
    where $g$ is whatever the input happens to be. In particular, all the
    outputs will now need to agree on the value of $u$ in addition to $v$,
    which will only be possible if there's exactly one circle in the square
    that all outputs assign to $v$.
\end{itemize}
\end{minisplit}

\dotbreak[Relative Definite (Haddock) Reading]

\begin{minisplit} % Haddock Definites
\begin{align*} % Haddock Deriv
  &
  \sv{\textrm{the circle in the square}} \\
  %
  %
  &
  \p{%
    \bitt{\textbf{the}_u\,\p{\l x \dt \hole}}{x}
    %
    \BSL{%
    %
    \p{%
      \bitt{\hole}{\textbf{circle}}
      %
      \MSL{%
      %
      \bitt{\hole}{\textbf{in}}
      %
      \FSL{%
      %
      \p{%
        \bitt{\textbf{the}_v\,\p{\l y \dt \hole}}{y}
        %
        \BSL{%
        %
        \bitt{\hole}{\textbf{square}}
        }
      }^{\dnar} } }
    } }
  }^{\dnar\,\reset} \\
  %
  %
  &
  \p{%
    \btower{\textbf{the}_u\,\p{\l x \dt \hole}, x}
    %
    \BSL{%
    %
    \p{%
      \btower{\hole, \textbf{circle}}
      %
      \MSL{%
      %
      \btower{\hole, \textbf{in}}
      %
      \FSL{%
      %
      \btower{%
        \l gh \dt
        \uset{\hole\,g^{v\mapsto y}\,\post{h}{\one_v} \giv \cn{sq}\,y}
      , y
      } } }
    } }
  }^{\dnar\,\reset} \\
  %
  %
  &
  \p{%
    \btower{%
      \textbf{the}_u\,\p{%
        \l xgh \dt
        \uset{\hole\,g^{v\mapsto y}\,\post{h}{\one_v} \giv \cn{sq}\,y}
      },
      \cn{circ}\,x \land \cn{in}\,y\,x
    }
  }^{\dnar\,\reset} \\
  %
  %
  &
  \p{%
    \btower{%
      \l gh \dt
      \uset{%
        \hole\,
        g^{\substack{u\mapsto x\\ v\mapsto y}}\,
        \post{\p{\post{h}{\one_u}}}{\one_v}
      \giv
        \cn{sq}\,y,\ \cn{circ}\,x,\ \cn{in}\,y\,x
      },
      x
    }
  }^{\reset} \\
  %
  %
  &
  \btower{%
    \l gh \col \abs{G_v} = \abs{G_u} = 1 \dt
    \uset{%
      \hole\,g^{\substack{u\mapsto x\\ v\mapsto y}}\,h
    \giv
      \cn{sq}\,y,\ \cn{circ}\,x,\ \cn{in}\,y\,x
    },
    x
  }
\end{align*}
%
\splitmini
%
\begin{itemize} % Haddock Notes
  \item
    The only difference here is that we do not reset the inner DP, which
    staves off its postsup until more information is accumulated in its scope.
  \item
    So when the outer DP is eventually reset, it ends up fixing the postsup of
    both definites to the same set of potential outputs.
  \item
    For any input $g$, the set of outputs $G$ that $\one_u$ and $\one_v$ will
    target is given below. All of the outputs in this set will map $u$ to some
    circle and $v$ to some square containing the circle they map to $u$.\\
    $G = \set{%
      \tup{x, g^{\substack{u\mapsto x\\ v\mapsto y}}, \post{\one_u}{\one_v}}
    \giv
      \cn{sq}\,x,\ \cn{circ}\,y,\ \cn{in}\,y\,x
    }$.
  \item
    Requiring that there be exactly one such $v$ and one such $u$ is
    tantamount to requiring that there be exactly one square \emph{that has a
    circle in it} and exactly one circle \emph{in that square}. In other
    words, there should be exactly one pair $\pair{x}{y}$ in $\cn{circ} \times
    \cn{sq}$ such that $\cn{in}\,y\,x$.
\end{itemize}
\end{minisplit}

\newpage
\dotbreak\vspace{-1em}

\section{Plurals}

\begin{minisplit} % Plurals
\begin{spreadlines}{0pt} % Plural Fragment
\begin{align*}
  \textbf{-s} &\coloneq
  \l Px \dt x \in \set{\mathlarger{\oplus} P' \giv P' \subseteq P} \\
  %
  \textbf{the}_u &\coloneq
  \l ckgh \dt
  \uset{%
    k\,x\,g' h'
  \giv
    \tup{\T,g',h'} \in c\,x\,g^{u\mapsto x}\,\post{h}{\one_u\circ\M_u}
  } \\
  %
  \M_u &\coloneq
  \l G \dt
  \set{%
    \tup{\cdot, g, \cdot} \in G
  \giv
    \neg\exists\tup{\cdot,g',\cdot} \in G \dt g'\,u \sqsupset g\,u
  } \\
  %
  \textbf{in} &\coloneq
  \l XY \dt
  \p{\forall x<X \dt \exists y<Y \dt \cn{in}\,x\,y} \land
  \p{\forall y<Y \dt \exists x<X \dt \cn{in}\,x\,y}
\end{align*}  
\end{spreadlines}
%
\splitmini
%
  \begin{itemize} % Plural Notes
    \item
      $\textbf{-s}$ is a boilerplate plural morpheme that builds sums from the
      atoms in its complement.
    \item
      $\M_u$ is a kind of maximization operator on outputs (Brasoveanu
      2012, Charlow 2014). It filters out those assignments in $g \in G$ that
      are strictly dominated, in the sense that they assign $u$ to a value
      that is a proper part of something assigned to $u$ by some other $g' \in
      G$.
    \item
      Predicates and relations are extended to sums in the usual cumulative
      way (Link, \dots)
  \end{itemize}
\end{minisplit}

\dotbreak[Plural Reset]

\begin{minisplit} % Plural Resets
\begin{align*} % Plural Reset Deriv
  \sv{\textrm{the squares}}
  %
  &\leadsto
  \p{%
    \btower{%
      \textbf{the}_v\,\p{\l Y \dt \hole},
      Y
    }
    %
    \BSL{%
    %
    \btower{%
      \hole,
      \textbf{square}
    }
    %
    \FSL{%
    %
    \btower{%
      \hole,
      \textbf{-s}
    } } }
  }^{\dnar\,\reset}
  %
  \leadsto
  \p{%
    \textbf{the}_v\,\p{\l Ygh \dt \set{\tup{\cn{sq's}\,Y, g, h}}}
  }^\reset \\
  %
  %
  &\leadsto
  \p{%
    \btower{%
      \l gh \dt
      \uset{%
        \hole\,g^{v\mapsto Y}\,\post{h}{\one_u\circ\M_v}
      \giv
        \cn{sq's}\,Y
      },
      Y
    }
  }^\reset \\
  %
  %
  &\leadsto
  \btower{%
    \l gh \col \abs{G_v} = 1 \dt
    \uset{%
      \hole\,g^{v\mapsto Y}\,h
    \giv
      \cn{sq}\,Y,\ \forall Z\col\cn{sq's} \dt Z \not\sqsupset Y
    },
    Y
  }
\end{align*}
%
\splitmini
%
\begin{itemize} % Reset Notes
  \item
    Once again, resetting $\sv{\textrm{the square}}$ in the last reduction
    step has no effect on its semantic shape, because it's still essentially
    $\sv{\textrm{a square}}$.
  \item
    But again, it fixes the constituent's postsup. For any incoming $g$, $G$
    will be equal to $\set{\tup{y, g^{v\mapsto Y}, \one_v\circ\M_v} \giv
    \cn{sq's}\,Y}$. The only output triples that will survive the $\M_v$
    filter will be those that are undominated in their choice of $v$, i.e.,
    those that assign $v$ to a sum that isn't part of any sum assigned to $v$
    by any other output. The subsequent $\one_v$ filter will then guarantee
    that there is exactly one such maximal referent across outputs (this
    should generally by trivial, given usual mereological assumptions).
\end{itemize} 
\end{minisplit}

\dotbreak[Absolute Plural Reading]

\begin{minisplit}[0.55] % Absolute Plurals
\begin{align*} % Absolute Plural Deriv
  &
  \sv{\textrm{the circles in the squares}} \\
  %
  %
  &
  \p{%
    \btower{\textbf{the}_u\,\p{\l X \dt \hole}, X}
    %
    \BSL{%
    %
    \p{%
      \btower{\hole, \textbf{circles}}
      %
      \MSL{%
      %
      \btower{\hole, \textbf{in}}
      %
      \FSL{%
      %
      \p{%
        \btower{\textbf{the}_v\,\p{\l Y \dt \hole}, Y}
        %
        \BSL{%
        %
        \btower{\hole, \textbf{squares}}
        }
      }^{\dnar\,\reset} } }
    } }
  }^{\dnar\,\reset} \\
  %
  %
  &
  \p{%
    \btower{\textbf{the}_u\,\p{\l X \dt \hole}, X}
    %
    \BSL{%
    %
    \p{%
      \btower{\hole, \textbf{circles}}
      %
      \MSL{%
      %
      \btower{\hole, \textbf{in}}
      %
      \FSL{%
      %
      \btower{%
        \l gh \col \abs{G_v} = 1 \dt
        \uset{%
          \hole\,g^{v\mapsto Y}\,h
        \giv
          \begin{array}{L}
            \cn{sq's}\,Y, \\
            \forall Z\col\cn{sq's} \dt Z \not\sqsupset Y
          \end{array}
        },
        Y
      } } }
    } }
  }^{\dnar\,\reset} \\
  %
  %
  &
  \p{%
    \btower{%
      \textbf{the}_u\,\p{%
        \l Xgh \col \abs{G_v} = 1 \dt
        \uset{%
          \hole\,g^{v\mapsto Y}\,h
        \giv
          \cn{sq}\,Y,\ \forall Z\col\cn{sq's} \dt Z \not\sqsupset Y
        }
      },
      \cn{circ's}\,X \land \cn{in}\,Y\,X
    }
  }^{\dnar\,\reset} \\
  %
  %
  &
  \p{%
    \btower{%
      \l gh \col \abs{G_v} = 1 \dt
      \uset{%
        \hole\,g^{\substack{u\mapsto X\\ v\mapsto Y}}
             \,\post{h}{\one_u\circ\M_u}
      \giv
        \begin{array}{L}
          \cn{sq's}\,Y,\ \cn{circ's}\,X,\ \cn{in}\,Y\,X, \\
          \forall Z\col\cn{sq's} \dt Z \not\sqsupset Y
        \end{array}
      },
      X
    }
  }^\reset \\
  %
  %
  &
  \btower{%
    \l gh \col \abs{G_v} = \abs{G'_u} = 1 \dt
    \uset{%
      \hole\,g^{\substack{u\mapsto X\\ v\mapsto Y}}\,h
    \giv
      \begin{array}{L}
        \cn{sq}\,Y,\ \cn{circ}\,X,\ \cn{in}\,Y\,X, \\
        \forall Z\col\cn{sq's} \dt Y \not\sqsubset Z, \\
        \forall Z \dt \cn{circ's}\,Z \land \cn{in}\,Y\,Z
                      \Rightarrow X \not\sqsubset Z
      \end{array}
    },
    X
  }
\end{align*}
%
\splitmini
%
\begin{itemize} % Absolute Plural Notes
  \item
    The inner definite is reset, freezing its supposition as above. $\M_v$
    will throw out any outputs that do not assign $v$ to the largest sum of
    circles, leaving $G = \set{\tup{Y, g^{v\mapsto Y}, \one_v\circ\M_v} \giv
    \cn{sq's}\,Y,\ \forall Z\col\cn{sq's} \dt Z \not\sqsupset Y}$. When the
    input assignment $g$ is eventually inserted, $\one_v$ will guarantee that
    this individual is constant across possible outputs (i.e., unique).
  \item
    When the host DP is reset, its maximality supposition $\M_u$ is applied to the
    set of outputs $ G' = \set{\tup{x, g^{\substack{u\mapsto X\\ v\mapsto
    Y}}, \one_u\circ\M_u} \giv \cn{sq's}\,Y,\ \cn{circ's}\,X,\ \cn{in}\,Y\,X,
    \ \forall Z\col\cn{sq's} \dt Z \not\sqsupset Y}$. This is the set out
    assignments that map $v$ to a (the) maximal sum of squares, and $v$ to a
    sum of circles that are (cumulatively) in that sum. The $\M_u$ filter will
    discard from this set any outputs that do not map $u$ to as big a sum as
    possible, i.e., the maximal sum of circles that are in $Y$. Finally
    $\one_u$ will guarantee that this maximal sum is unique.
  \item
    For this whole update to be defined, it must be the case that every square
    has at least one circle in it.
\end{itemize}
\end{minisplit}

\newpage

\dotbreak[Relative Plural (Haddock) Reading]

\begin{minisplit} % Plural Haddock Definites
\begin{align*} % Plural Haddock Deriv
  &
  \sv{\textrm{the circles in the squares}} \\
  %
  %
  &
  \p{%
    \bitt{\textbf{the}_u\,\p{\l X \dt \hole}}{X}
    %
    \BSL{%
    %
    \p{%
      \bitt{\hole}{\textbf{circles}}
      %
      \MSL{%
      %
      \bitt{\hole}{\textbf{in}}
      %
      \FSL{%
      %
      \p{%
        \bitt{\textbf{the}_v\,\p{\l Y \dt \hole}}{Y}
        %
        \BSL{%
        %
        \bitt{\hole}{\textbf{squares}}
        }
      }^{\dnar} } }
    } }
  }^{\dnar\,\reset} \\
  %
  %
  &
  \p{%
    \btower{\textbf{the}_u\,\p{\l X \dt \hole}, X}
    %
    \BSL{%
    %
    \p{%
      \btower{\hole, \textbf{circles}}
      %
      \MSL{%
      %
      \btower{\hole, \textbf{in}}
      %
      \FSL{%
      %
      \btower{%
        \l gh \dt
        \uset{\hole\,g^{v\mapsto Y}\,\post{h}{\one_v\circ\M_v} \giv \cn{sq's}\,Y}
      , y
      } } }
    } }
  }^{\dnar\,\reset} \\
  %
  %
  &
  \p{%
    \btower{%
      \textbf{the}_u\,\p{%
        \l Xgh \dt
        \uset{\hole\,g^{v\mapsto Y}\,\post{h}{\one_v\circ\M_v} \giv \cn{sq's}\,Y}
      },
      \cn{circ's}\,X \land \cn{in}\,Y\,X
    }
  }^{\dnar\,\reset} \\
  %
  %
  &
  \p{%
    \btower{%
      \l gh \dt
      \uset{%
        \hole\,
        g^{\substack{u\mapsto X\\ v\mapsto Y}}\,
        \post{\p{\post{h}{\one_u\circ\M_u}}}{\one_v\circ\M_v}
      \giv
        \cn{sq's}\,Y,\ \cn{circ's}\,X,\ \cn{in}\,Y\,X
      },
      X
    }
  }^{\reset} \\
  %
  %
  &
  \btower{%
    \l gh \col \abs{G_v} = \abs{G_u} = 1 \dt
    \uset{%
      \hole\,g^{\substack{u\mapsto X\\ v\mapsto Y}}\,h
    \giv
      \cn{sq's}\,Y,\ \cn{circ's}\,X,\ \cn{in}\,Y\,X
    },
    X
  }
\end{align*}

\vspace{-1.2em}
\parbox{0.95\textwidth}{%
\begin{itemize}
  \item
    This double-barreled cumulation and concomitant cardinatlity checking is
    exactly as (is intended) in Brasoveanu 2012. This brings definites in line
    with other cardinality post-supposing expressions like `exactly three'.
\end{itemize}
}
%
\splitmini
%
\begin{itemize} % Plural Haddock Notes
  \item
    As before, the only difference here is that we do not reset the inner DP,
    which staves off its postsup until more information is accumulated in its
    scope. So when the outer DP is eventually reset, it ends up fixing the
    postsup of both definites to the same set of potential outputs.
  \item
    For any input $g$, the set of outputs $G$ that
    $\one_u\circ\M_u\circ\one_v\circ\M_v$ will
    target is given below. All of the outputs in this set will map $u$ to some
    plurality of circles and $v$ to some plurality of squares (cumulatively)
    containing those circles.\\
    $G = \set{%
      \tup{x, g^{\substack{u\mapsto X\\ v\mapsto Y}},
             \post{\one_u\circ\M_u}{\one_v\circ\M_v}}
    \giv
      \cn{sq's}\,X,\ \cn{circ's}\,Y,\ \cn{in}\,Y\,X
    }$.
  \item
    $\M_v$ will throw out any candidate output that is dominated by another in
    its choice for $v$, and $\one_v$ will guarantee (trivially, as usual) that
    there is exactly one such undominated candidate. That is, $\M_v$ will
    filter out any outputs that do not assign $v$ to the maximal sum of
    circle-containing squares.
  \item
    Likewise, $\M_u$ will then throw out any candidates that are dominated in
    their choice for $u$; that is, any candidates that do not assign $u$ to
    the maximal sum of circles that are contained in the sum total of
    circle-containing squares. So by the end, we'll have a set of outputs all
    of which assign $v$ to the totality of squares that contain (\emph{any}
    circles) and $u$ to the totality of circles that are in (\emph{any})
    squares.
\end{itemize}
\end{minisplit}

\dotbreak\vspace{-1em}

\section{DP-Internal Superlatives}

\begin{minisplit} % DP Superlatives
\begin{spreadlines}{0pt} % DP Sup Fragment
\begin{align*}
  \textbf{older} &\coloneq
  \l xy \dt \cn{age}\,x > \cn{age}\,y \\
  %
  \textbf{the}_u &\coloneq
  \l \mathcal{M}ckgh \dt
  \uset{%
    k\,x\,g'\,h'
  \giv
    \tup{\T,g',h'} \in c\,x\,g^{u\mapsto x}\,\post{h}{\one_u\circ\mathcal{M}}
  } \\
  %
  \M_u^f &\coloneq
  \l G \dt
  \set{%
    \tup{\cdot,g,\cdot} \in G
  \giv
    \neg\exists\tup{\cdot,g',\cdot} \in G \dt f\,(g'\,u)\,(g\,u)
  } \\
  %
  \textbf{est}_u &\coloneq
  \l f \dt \M_u^f \\
  %
  \textbf{oldest}_u &=
  \textbf{est}_u\,\textbf{older} =
  \l G \dt
  \set{%
    \tup{\cdot,g,\cdot} \in G
  \giv
    \neg\exists\tup{\cdot,g',\cdot} \in G \dt
    \cn{age}\,(g'\,u) > \cn{age}\,(g\,u)
  }
\end{align*}
\end{spreadlines}
%
\splitmini
%
\begin{itemize} % DP Sup Notes
  \item
    $\textbf{est}$ abstracts over the ordering function that $\M$ uses to
    compare individuals. In the case of pluralities, $\M_u$ filters away any
    output that assigns $u$ to a \emph{smaller sum} than it could have (ie., a
    smaller sum than one of the other outputs assigns to $u$). In the case of
    $\textbf{oldest}$, $\M_u$ filters outputs that assign $u$ to a
    \emph{younger} individual than they could have.
  \item
    And now here's the swim move: $\textbf{the}$ absorbs the max operator and
    then carries it along for the ride. ``Absolute'' definite become absolute
    superlatives; ``Haddock'' definites become relative superlatives. This
    part is my favorite.
\end{itemize}
\end{minisplit}

\dotbreak[Superlative Reset]

\begin{minisplit} % DP Superlative Reset
\begin{align*} % DP Sup Reset Deriv
  &
  \sv{\textrm{the oldest squirrel}} \\
  %
  %
  &\leadsto
  \p{%
    \btower{\textbf{the}_v\,\M_v^\cn{ag}\,\p{\l y \dt \hole}, y}
    %
    \BSL{%
    %
    \btower{\hole, \textbf{squirrel}}
    }
  }^{\dnar\,\reset}
  %
  \leadsto
  \p{%
    \textbf{the}_v\,\M_v^\cn{ag}\,\p{\l ygh \dt \set{\tup{\cn{sq}\,y,g,h}}}
  }^\reset \\
  %
  %
  &\leadsto
  \p{%
    \btower{%
      \l gh \dt
      \uset{%
        \hole\,g^{v\mapsto y}\,\post{h}{\one_v\circ\M_v^\cn{ag}}
      \giv
        \cn{sq}\,y
      }
    , y
    }
  }^\reset \\
  %
  %
  &\leadsto
  \btower{%
    \l gh \col \abs{G_v} = 1 \dt
    \uset{%
      \hole\,g^{v\mapsto y}\,h
    \giv
      \cn{sq}\,y,\  \forall z\col\cn{sq} \dt \neg\,\cn{older}\,z\,y
    },
    y
  }
\end{align*}
%
\splitmini
%
\begin{itemize} % DP Sup Reset Notes
  \item
    Resetting the superlative DP has two effects: (1) it fixes the set of
    individuals that the superlative operator $\M$ compares, in this case
    squirrels with respect to age; and (2) it freezes the postsup associated
    with the definite article.
  \item
    Here $G = \set{\tup{y,g^{v\mapsto y},\one_v\circ\M_v} \giv \cn{sq}\,y,\
    \forall z\col\cn{sq} \dt \neg\,\cn{older}\,z\,y}$. This set could in
    principle have multiple winners (multiple squirrels with the same greatest
    age), but the definite determiner rules that out; all the $g \in G$ need
    to point $v$ to \emph{the same} squirrel.
\end{itemize}
\end{minisplit}

\newpage
\dotbreak[Absolute DP-Internal Superlatives]

\begin{minisplit}[0.55] % DP Superlatives, Absolute
\begin{align*} % Absolute DP Sup Deriv
  &
  \sv{\textrm{the circus with the oldest squirrel}} \\
  %
  %
  &
  \p{%
    \bitt{\textbf{the}_u\,\p{\l x \dt \hole}}{x}
    %
    \BSL{%
    %
    \p{%
      \bitt{\hole}{\textbf{circus}}
      %
      \MSL{%
      %
      \bitt{\hole}{\textbf{with}}
      %
      \FSL{%
      %
      \p{%
        \bitt{\textbf{the}_v\,\M_v^\cn{ag}\,\p{\l y \dt \hole}}{y}
        %
        \BSL{%
        %
        \bitt{\hole}{\textbf{squirrel}}
        }
      }^{\dnar\,\reset} } }
    } }
  }^{\dnar\,\reset} \\
  %
  %
  &
  \p{%
    \bitt{\textbf{the}_u\,\p{\l x \dt \hole}}{x}
    %
    \BSL{%
    %
    \p{%
      \bitt{\hole}{\textbf{circus}}
      %
      \MSL{%
      %
      \bitt{\hole}{\textbf{with}}
      %
      \FSL{%
      %
      \bitt{%
        \l gh \col |G_v| = 1 \dt
          \uset{%
            \hole\,g^{v\mapsto y}\,h
          \giv
            \begin{array}{L}
              \cn{sq}\,y,\\ \forall z\col\cn{sq} \dt \neg\,\cn{older}\,z\,y
            \end{array}
          }
      }{y}
      } }
    } }
  }^{\dnar\,\reset} \\
  %
  %
  &
  \p{%
    \bitt{%
      \textbf{the}_u\,\p{%
        \l xgh \col |G_v| = 1 \dt
          \uset{%
            \hole\,g^{v\mapsto y}\,h
          \giv
            \cn{sq}\,y,\
            \forall z\col\cn{sq} \dt \neg\,\cn{older}\,z\,y
          }
      }
    }{%
      \cn{circ}\,x \land \cn{with}\,y\,x
    }
  }^{\dnar\,\reset} \\
  %
  %
  &
  \p{%
    \bitt{%
      \l gh \col |G_v| = 1 \dt
        \uset{%
          \hole\,g^{\substack{u\mapsto x\\ v\mapsto y}}\,\post{h}{\one_u}
        \giv
          \begin{array}{L}
          \cn{sq}\,y,\ \cn{circ}\,x,\ \cn{with}\,y\,x, \\
          \forall z\col\cn{sq} \dt \neg\,\cn{older}\,z\,y
          \end{array}
        }
    }{x}
  }^\reset \\
  %
  %
  &
  \bitt{%
    \l gh \col |G'_u| = |G_v| = 1 \dt
      \uset{%
        \hole\,g^{\substack{u\mapsto x\\ v\mapsto y}}\,h
      \giv
        \begin{array}{L}
        \cn{sq}\,y,\ \cn{circ}\,x,\ \cn{with}\,y\,x, \\
        \forall z\col\cn{sq} \dt \neg\,\cn{older}\,z\,y
        \end{array}
      }
  }{x}
\end{align*}
%
\splitmini
%
\begin{itemize} % Absolute DP Sup Notes
  \item
    When the inner DP is reset, it has the effect just demonstrated: the set
    of outputs is restricted to those that map $v$ to the (unique) oldest
    squirrel.
  \item
    The outer definite has no explicit maximization operator. For
    simplicity, for now, I assume that that the $\mathcal{M}$ argument to
    $\textbf{the}$ is optional. But even max-less, the definite still imposes
    a uniquness condition on the values assigned to $u$.
  \item
    Altogether, after the host DP has been reset, we will have an
    essentially nondeterministic (continuized) individual with two
    presuppositions: (1) there is exactly one squirrel with no elders; and (2)
    there is exactly one circus that he is with.
\end{itemize}
\end{minisplit}

\dotbreak[Relative DP-Internal Superlatives]

\begin{minisplit} % DP Superlative, Relative
\begin{align*} % Relative DP Sup Deriv
  &
  \sv{\textrm{the circus with the oldest squirrel}} \\
  %
  %
  &
  \p{%
    \bitt{\textbf{the}_u\,\p{\l x \dt \hole}}{x}
    %
    \BSL{%
    %
    \p{%
      \bitt{\hole}{\textbf{circus}}
      %
      \MSL{%
      %
      \bitt{\hole}{\textbf{with}}
      %
      \FSL{%
      %
      \p{%
        \bitt{\textbf{the}_v\,\M_v^\cn{ag}\,\p{\l y \dt \hole}}{y}
        %
        \BSL{%
        %
        \bitt{\hole}{\textbf{squirrel}}
        }
      }^{\dnar} } }
    } }
  }^{\dnar\,\reset} \\
  %
  %
  &
  \p{%
    \bitt{\textbf{the}_u\,\p{\l x \dt \hole}}{x}
    %
    \BSL{%
    %
    \p{%
      \bitt{\hole}{\textbf{circus}}
      %
      \MSL{%
      %
      \bitt{\hole}{\textbf{with}}
      %
      \FSL{%
      %
      \bitt{%
        \l gh \dt
        \uset{%
          \hole\,g^{v\mapsto y}\post{h}{\one_v\circ\M_v^\cn{ag}}
        \giv
          \cn{sq}\,y
        }
      }{y}
      } }
    } }
  }^{\dnar\,\reset} \\
  %
  %
  &
  \p{%
    \bitt{%
      \textbf{the}_u\,\p{%
        \l xgh \dt
        \uset{%
          \hole\,g^{v\mapsto y}\post{h}{\one_v\circ\M_v^\cn{ag}}
        \giv
          \cn{sq}\,y
        }
      }
    }{%
      \cn{circ}\,x \land \cn{with}\,y\,x
    }
  }^{\dnar\,\reset} \\
  %
  %
  &
  \p{%
    \bitt{%
      \l gh \dt
        \uset{%
          \hole\,g^{\substack{u\mapsto x\\ v\mapsto y}}
               \,\post{\p{\post{h}{\one_u}}}{\one_v\circ\M_v}
        \giv
          \cn{sq}\,y,\ \cn{circ}\,x,\ \cn{with}\,y\,x
        }
    }{x}
  }^{\reset} \\
  %
  %
  &
  \tower{%
    \l gh \col |G_u| = |G_v| = 1 \dt
    \uset{%
      \hole\,g^{\substack{u\mapsto x\\ v\mapsto y}}\,h
    \giv
      \begin{array}{L}
        \cn{sq}\,y,\ \cn{circ}\,x,\ \cn{with}\,y\,x\\
        \forall z \col \cn{sq} \dt
        \p{\exists a \dt \cn{with}\,z\,a} \Rightarrow \neg\,\cn{older}\,z\,a
      \end{array}
    },
    x
  }
\end{align*}
%
\splitmini
%
\begin{itemize} % Rel DP Sup Notes
  \item
    The only difference here is that we do not reset the inner DP, which
    staves off its postsup until more of the context is accumulated in
    its scope. By the time the postsup is actually evaluated, all of the
    potential output assignments will map $v$ to squirrles \emph{that are in
    some circus}.
  \item
    Resetting the outer DP fixes the suppositions of \emph{both} definites,
    bolting them down to the same set of outputs.
  \item
    For any input $g$, $G =
    \set{%
      \tup{%
        x
      , g^{\substack{u\mapsto x\\ v\mapsto y}}
      , \post{\one_u}{\one_v\circ\M_v}
      }
    \giv
      \cn{sq}\,x,\ \cn{circ}\,y,\ \cn{with}\,y\,x
    }$
    is the set out outputs that assign $v$ to a squirrel and $v$ to some
    circus employing it.
  \item
    Before either uniqueness check is enforced, $\M_v$ will filter out any
    assignments that are dominated $v$-wise --- that is, any assignments that
    send $v$ to a younger circus squirrel than they could have.
  \item
    Then, $\one_v$ will guarantee that there is exactly one such $v$ across
    all the surviving $g$s, in effect guaranteeing that there is exactly one
    circus squirrel that is older than all the others. Subsequently, $\one_u$
    will enforce that squirrel's no-compete clause with his particular circus.
    Altogether, this will ensure that one circus squirrel is older than all
    the others, and that he works for exactly one circus.
\end{itemize}
\end{minisplit}

\newpage
\dotbreak\vspace{-1em}

\section{Focus}

\begin{minisplit} % Adding Focus
\begin{spreadlines}{0em} % Focus fragment
\begin{align*}
  \M_u^f &=
  \lambda G \dt
    \set{%
      \tup{\cn{\T}, g, h}
    \giv
      \tup{\tup{\cn{\T}, \cdot}, g, h} \in G
      ,\ \, 
      \neg\exists\tup{\tup{\cdot, \beta}, g', \cdot} \in G \dt
        \ms{\bigvee}\!\beta \land f\,(g\,u)\,(g'\,u)
    } \\
  %
  % \textbf{older} &=
  % \lambda xy \dt \cn{age}\,x < \cn{age}\,y \\
  % %
  % \textbf{est}_u &=
  % \lambda f \dt \M_u^f\\
  %
  \textbf{oldest}_u &=
  \textbf{est}_u\,\textbf{older} \\
  &=
  \lambda G \dt
    \set{%
      \tup{\T, g, h}
    \giv
      \begin{array}{L}
      \tup{\tup{\T,\cdot}, g, h} \in G,\\
      \neg\exists\tup{\tup{\cdot, \beta}, g', \cdot} \in G \dt
        \bigvee\!\!\beta \land \cn{age}\,(g\,u) < \cn{age}\,(g'\,u)
      \end{array}
    }
  %
  % \textbf{the}_u &=
  % \lambda\mathcal{M}ckgh \dt
  % \uset{%
  %   k\,x\,g'\,h'
  % \giv
  %   x\in \mathcal{D}_e,\
  %   \tup{\T,g',h'} \in c\,x\,g^{u\mapsto x}\,\post{h}{\one_u\circ\mathcal{M}}
  % }
\end{align*}
\end{spreadlines}

\vspace{-1.2em}
\parbox{0.95\textwidth}{%
\begin{itemize}
  \item
    Note that $\textbf{the}$ is exactly as before. It is neutral between
    focus-sensitive and focus-insensitive ``maximizers''.
\end{itemize}
}
%
\splitmini
%
\begin{itemize} % Focus notes
  \item
    Superlatives are apparently focus-sensitive. \objl{John sold
    MARY the most expensive painting} means he sold Mary a more expensive
    painting than he sold anyone else.
  \item
    To prepare for this, the entry for $\textbf{est}$ here expects to see
    outputs with focus-pairs. It filters out the assignments
    that either fail to satisfy the ``ordinary'' component of their prejacent,
    or fail to assign $u$ to a value whose $f$-degree is greater than that of
    every other acceptable assignment's choice of $u$.
\end{itemize}
\end{minisplit}

\dotbreak[Derivations]

\begin{minisplit} % Focus Derivs
\begin{align*}
  &
  \sv{\textrm{John drew the oldest square}} \\
  %
  %
  &
  \p{%
  \btower{%
    \p{\eta\,\cn{j}}^\rhd \bind \p{\l j \dt \hole},
    j^\F \star \p{\l x \dt \hole},
    x
  }
  %
  \FSL{%
  \btower{%
    \hole,
    \hole,
    \cn{drew}
  } }
  %
  \BSL{%
  \btower{%
    \lambda gh \dt
    \uset{%
      \hole\,g^{v\mapsto y}\,\post{h}{\one_u\circ\M_v^\cn{ag}}
    \giv
      \cn{sq}\,y
    },
    \hole,
    y
  } } }^\reset \\
  %
  %
  &
  \p{%
  \btower{%
    \l gh \dt
    \uset{%
      \hole\,g^{\substack{u\mapsto \cn{j}\\ v\mapsto y}}
           \,\post{h}{\one_u\circ\M_v^\cn{ag}}
    \giv
      \cn{sq}\,y
    },
    \cn{j}^\F \bind \p{\l x \dt \hole},
    \cn{drew}\,y\,x
  } }^\reset \\
  %
  %
  &
  \btower{%
    \l gh \col |G_u| = 1 \dt
    \uset{%
      \hole\,g^{\substack{u\mapsto\cn{j}\\ v\mapsto y}}\,h
    \giv
      \begin{array}{L}
        \cn{sq}\,y,\ \
        \cn{drew}\,y\,\cn{j},\\
        \forall z \col \cn{sq} \dt
        \p{\exists a \dt \cn{drew}\,z\,a} \Rightarrow \neg\,\cn{older}\,z\,y
      \end{array}
    } 
  }
\end{align*}  
%
\splitmini
%
\begin{itemize}
  \item
    The superlative needs to outscope focus so that it can filter out squares
    drawn by John's alternatives. This is effectively guaranteed by stashing
    the superlative operator in the postsuppositional cubby, but we want it to
    out-tower focus, just so the reset operator works out (though this can
    probably be generalized).
  \item
    Here for the first time, we see the index on $\textbf{the}$ coming apart
    from that on $\M$. This is because, intuitively, to me anyway, the
    \objl{the} of relative \objl{the oldest} targets \emph{John}, not the
    squares that people have drawn. It says that John is \emph{the winner} of
    the square-drawing contest, regardless of how many winning squares he has
    drawn.
  \item
    $\M_v$ keeps all of the $g$s that map $v$ to a square that (i) John drew,
    and (ii) is at least as big as any square that anybody drew. If there
    aren't any such squares, the sentence is false. If there are any such
    squares, the sentence is guaranteed to succeed because $\one_u$ can't
    fail. It checks that all of the outputs map $u$ to the same individual,
    but of course they do; they all map $u$ to John (by design).
\end{itemize}
\end{minisplit}

\dotbreak

\begin{minipage}[t]{0.5\textwidth} % Focus-State Monad Fragment
\begin{spreadlines}{0pt}
\begin{align*}
  \mathcal{F}\alpha
  &\coloneq
  \sigma \sto \typepair{%
    \set{\typepair{\alpha}{\sigma}}
  }{%
    \set{\typepair{\alpha}{\sigma}}
  } \\
  %
  \eta\,x
  &\coloneq
  \l g \dt \pair{\set{\pair{x}{g}}}{\set{\pair{x}{g}}} \\
  %
  m \bind f
  &\coloneq
  \l g \dt
    \thickpair{%
      \uset{\p{f\,x\,g'}_1 \giv \pair{x}{g'} \in \p{m\,g}_1}
    }{%
      \uset{\p{f\,x\,g'}_2 \giv \pair{x}{g'} \in \p{m\,g}_2}
    }
\end{align*}
\end{spreadlines}
\end{minipage}
%
%
\begin{minipage}[t]{0.5\textwidth} % Focus-State Monad Notes
\begin{itemize}
  \item
    Rolling focus and state simultaneously to grease the wheels
\end{itemize}
\end{minipage}

\dotbreak\vspace{-1em}

\section{Adjectival Exclusives}

\begin{minipage}[t]{0.5\textwidth} % Only Fragment
% \begin{spreadlines}{0pt}
\begin{align*}
  \cn{true} &\coloneq
  \l G \dt \bigvee\set{\alpha \giv \pair{\alpha}{g} \in G} \\
  %
  \textbf{not} &\coloneq
  \l mg \dt \eta\,\p{\neg \p{m\,g}}\,g \\
  %
  \textbf{only}_u &\coloneq 
  \l G \col \cn{true}\,G \dt \M_u\,G \\
  %
  \textbf{the}_u &\coloneq
  \l \mathcal{M}ckg \dt |G_u| = 1 \dt G, \\
  &\hphantom{{}\coloneq{}}
    \text{where}\ 
    G = \mathcal{M}_u\uset{%
      k\,x\,g'
    \giv
      \pair{\cn{T}}{g'} \in c\,x\,g^{u\mapsto x}
    }
\end{align*}
% \end{spreadlines}
\end{minipage}
%
%
\begin{minipage}[t]{0.5\textwidth} % Only Fragment Notes
\begin{spreadlines}{0pt}
\begin{itemize}
  \item
    $\cn{true}$ and $\textbf{not}$ are bog standard dynamic booleans: an
    update is ``true'' iff it generates at least one successful output
  \item
    Adjectival $\textbf{only}$ is just $\M$ plus a presupposition (cf.\
    Coppock and Beaver 2012, 2014)!
  \item
    So here's the swim move: feed $\textbf{only}$ into $\textbf{the}$, and
    then let the scope of the exclusivity ride the scope of the determiner.
    This will predict relative readings for adj $\textbf{only}$.
\end{itemize}
\end{spreadlines}
\end{minipage}

\dotbreak

% John sold the only cars \\

% Anna didn't give the only good talk




\end{document}
