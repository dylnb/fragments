\documentclass[10pt,fleqn]{article}


%%%%% deal with the excessive number of packages %%%%%
\usepackage{etex}

\usepackage{silence} % suppres font shape warnings
  \WarningFilter{latexfont}{Font shape}
\usepackage{ifthen} % for conditional macros

%%%%% general math %%%%%
\usepackage[tbtags]{mathtools} % loads amsmath
\usepackage{amssymb,stmaryrd}
  % \SetSymbolFont{stmry}{bold}{U}{stmry}{m}{n}
\usepackage{relsize} % change size of math operators
\usepackage{exscale} % change size of math operators arbitrarily
\usepackage{scalerel} % change size of math delimiters
% prevent align env at top of minipage from adding additional padding
\usepackage{etoolbox}
  \makeatletter
  \pretocmd\start@align{%
    \if@minipage\kern-\topskip\kern-\abovedisplayskip\fi
  }{}{}
  \makeatother

%%% all-purpose math macros
% parens
\DeclarePairedDelimiterX\PARENS[1](){#1}
\newcommand{\p}[1]{\PARENS*{#1}}
% set comprehension: \set{ ... \giv ... } = { ... | ... }
\providecommand{\giv}{}
\DeclarePairedDelimiterX\SET[1]\{\}{%
  \renewcommand{\giv}{\nonscript\:\delimsize\vert\nonscript\:\mathopen{}}
  #1
}
\newcommand{\set}[1]{\SET*{#1}}
% grand union: \uset{ ... \giv ... } = U{ ... | ... }
\DeclarePairedDelimiterXPP\USET[1]{\!\bigcup}\{\}{}{%
  \renewcommand{\giv}{\nonscript\:\delimsize\vert\nonscript\:\mathopen{}}
  #1
}
\newcommand{\uset}[1]{\USET*{#1}}
% tuples
\DeclarePairedDelimiterX\TUP[1]\langle\rangle{#1}
% tuple of arbitrary length
\makeatletter
\newcommand{\tup}[1]{%
  \ensuremath{%
    \TUP*{\my@tups #1,\relax\noexpand\@eolst}%
  }%
}
\def\my@tups #1,#2\@eolst{%
  \ifx\relax#2\relax
    #1%
  \else
    #1,\pt \my@tups #2\@eolst%
  \fi
}
\makeatother
\newcommand{\pair}[2]{\tup{#1, #2}}
\newcommand{\thickpair}[2]{\TUP*{#1,\ \ #2}}
\newcommand{\typepair}[2]{#1 \ast #2}
% verts
\DeclarePairedDelimiterX\ABS[1]||{#1}
\newcommand{\abs}[1]{\ABS*{#1}}

%%%%% examples, lists, footnotes, citations %%%%%
\usepackage[bottom,multiple]{footmisc} % force footnotes to bottom of page
\usepackage{enumitem} % customize lists
\usepackage{epltxfn} % expex examples in footnotes
\usepackage{expex} % for example sentences
\newcommand{\excite}[3][]{% citation at right edge of example
  \ifthenelse{\equal{#1}{}}{% no opt arg (no prefix) 
    \ifthenelse{\equal{#3}{ibid}}{% no reference to look up
      \rightcomment{[\textit{ibid}:~#2]}
    }{% look up reference
      \rightcomment{[\citealt[#2]{#3}]}
    }
  }{% optional argument present
    \ifthenelse{\equal{#1}{citealias}}{% use citealias
      \rightcomment{[\citetalias{#3}:~#2]}
    }{% set opt arg as citation prefix
      \ifthenelse{\equal{#3}{ibid}}{% no reference to look up
        \rightcomment{[#1 \textit{ibid}:~#2]}
      }{% look up reference
        \rightcomment{[#1 \citealt[#2]{#3}]}
      }
    }
  }
}
% inline citation style abbreviations
\usepackage{natbib}
  \bibpunct[: ]{(}{)}{,}{a}{}{,}
  \newcommand{\BIBand}{\&}
  \setlength{\bibsep}{0pt}
  \setlength{\bibhang}{0.25in}
  \bibliographystyle{sp}
  \newcommand{\posscitet}[1]{\citeauthor{#1}'s (\citeyear{#1})}
  \newcommand{\possciteauthor}[1]{\citeauthor{#1}'s}
  \newcommand{\pgposscitet}[2]{\citeauthor{#1}'s (\citeyear{#1}:~#2)}
  \newcommand{\secposscitet}[2]{\citeauthor{#1}'s (\citeyear{#1}:~$\S$#2)}
  \newcommand{\pgcitealt}[2]{\citealt{#1}:~#2}
  \newcommand{\seccitealt}[2]{\citealt{#1}:~$\S$#2}
  \newcommand{\pgcitep}[2]{(\citealt{#1}:~#2)}
  \newcommand{\seccitep}[2]{(\citealt{#1}:~$\S$#2)}
  \newcommand{\pgcitet}[2]{\citeauthor{#1} (\citeyear{#1}:~#2)}
  \newcommand{\seccitet}[2]{\citeauthor{#1} (\citeyear{#1}:~$\S$#2)}



%%%%% tables %%%%%
\usepackage{array} % more control over column styles
\usepackage{booktabs} % better rules

%%% array columns without spaces
\newcolumntype{L}{@{}l@{}}
\newcolumntype{R}{@{}r@{}}
\newcolumntype{C}{@{}c@{}}



%%%%% layout and formatting %%%%%
\usepackage[margin=1in]{geometry} % margins
\usepackage{sectsty} % customize section title appearance
  \allsectionsfont{\normalsize}

%%% customize the abstract
\newcommand{\frontmatterspacing}[1]{%
  \small
  \topsep 10pt
  \advance\topsep by 3.5ex plus -1ex minus -0.2ex
  \setlength{\listparindent}{0em}
  \setlength{\itemindent}{0em}
  \setlength{\leftmargin}{#1}
  \setlength{\rightmargin}{\leftmargin}
  \setlength{\parskip}{0em}
}
\renewenvironment{abstract}{%
  \list{}{\frontmatterspacing{0.25in}}%
  \item\relax\textbf{\abstractname}
}{\endlist}

%%% font preferences
\usepackage[T1]{fontenc}
\usepackage[utf8]{inputenc}
\usepackage{libertine}
\usepackage[libertine,liby,bigdelims]{newtxmath}

%%% links
\RequirePackage[usenames]{xcolor}
\definecolor{splinkcolor}{rgb}{.0,.2,.4}
\RequirePackage[colorlinks,breaklinks,
                linkcolor=splinkcolor, 
                urlcolor=splinkcolor, 
                citecolor=splinkcolor,
                filecolor=splinkcolor,
                plainpages=false,
                pdfpagelabels,
                bookmarks=false,
                pdfstartview=FitH]{hyperref}
\newcommand{\doi}[1]{\url{http://dx.doi.org/#1}}
\urlstyle{rm}
\usepackage{hyperref}


%%%%% additional symbols and macros %%%%%%

%%% sp macros
\newcommand{\sv}[1]{\llbracket #1 \rrbracket}
\newcommand{\with}{\mathbin{\&}}

%%% basic convenience macros
\newcommand{\pt}{\hspace{1pt}}
\newcommand{\ppt}{\hspace{2pt}}
\newcommand{\col}{\pt{:}\ppt}
\newcommand{\dt}{\pt{.}\ppt}
\newcommand{\cn}[1]{{\sf #1}}
\newcommand{\cat}{\hspace{-2pt}\cdot\hspace{-2pt}}
\newcommand{\sto}{\mathbin{\shortrightarrow}}
\newcommand{\must}{\mathlarger{\mathlarger{\Box}}}
\newcommand{\rest}[2]{#1 \col #2}
\newcommand{\objl}[1]{`#1'}
\newcommand{\tru}{\textsf{\bfseries T}}
\newcommand{\fals}{\textsf{\bfseries F}}
\newcommand{\ms}[1]{\mathsmaller{#1}}

% phantom to width of second argument
\newcommand*\phantomas[3][c]{%
   \ifmmode
     \makebox[\widthof{$#2$}][#1]{$#3$}%
   \else
     \makebox[\widthof{#2}][#1]{#3}%
   \fi
}
% dots to break sections in handouts, etc
\newcommand{\dotbreak}[1][]{%
  % \vspace{0.5em}\dotfill\hspace{0.5\textwidth} \\ \textsc{#1}\vspace{0.5em}
  \ifthenelse{\equal{#1}{}}{%
    \vspace{0.5em}\dotfill\vspace{0.5em}
  }{%
    \vspace{0.5em}\dotfill\hspace{0.5\textwidth} \\ \textsc{#1}\vspace{0.5em}
  }
}
% two-column minipage, with widths controlled by optional param
\newenvironment{minisplit}[1][0.5]{%
  \newcommand{\splitmini}{%
    \end{minipage}%
    \begin{minipage}[t]{\dimexpr\textwidth-#1\textwidth\relax}%
  }
  \noindent\ignorespaces%
  \begin{minipage}[t]{#1\textwidth}%
}{%
  \end{minipage}%
  \ignorespacesafterend%
}

%%% judgment diacritics
\newcommand{\att}{${}^{\boldsymbol\gamma}$}
\newcommand{\yes}{\textsuperscript{\checkmark}}
\newcommand{\bad}{\textsuperscript{\small\#}}
\newcommand{\mar}{\textsuperscript{?}}
\newcommand{\ung}{*}

%%% monads and towers
\renewcommand{\l}{\lambda}
\newcommand{\mtype}[2]{\mathbb{#1}_{#2}}
\newcommand{\bind}{\star}
\newcommand{\unit}{\eta}
\newcommand{\nil}{\varepsilon}
\newcommand{\dnar}{\downarrow}
\newcommand{\upar}{\uparrow}
\newcommand{\reset}{\downupharpoons}
\newcommand{\hole}{[\,]}
\newcommand{\fsl}{\sslash}
\newcommand{\bsl}{\bbslash}
\newcommand{\msl}{\,\|\,}
\newcommand{\FSL}[1]{\ \stretchrel[600]{\ \fsl\ }{#1}}
\newcommand{\BSL}[1]{\ \stretchrel[600]{\ \bsl\ }{#1}}
\newcommand{\MSL}[1]{\ \stretchrel[600]{\ \msl\ }{#1}}

\makeatletter
\newcommand{\tower}[2][\my@btows]{%
  \ensuremath{\mathinner{%
      \begin{tabular}[c]{@{}>{$\displaystyle}c<{$}@{}}%
        #1 #2,\relax\noexpand\@eolst%
      \end{tabular}%
  }}%
}
\def\my@ttows #1,#2,#3\@eolst{%
  \ifx\relax#3\relax
    \strut\ensuremath{#1}\\
  \else
    \strut\ensuremath{#1}\ \hfil\vrule\hfil\ \ensuremath{#2}\\
    \hline
    \my@ttows #3\@eolst%
  \fi
}
\def\my@btows #1,#2\@eolst{%
  \ifx\relax#2\relax
    \ensuremath{#1}%
  \else
    \ensuremath{#1}\\\hline%
    \my@btows #2\@eolst%
  \fi
}
% tripartite tower of aribitrary depth
\newcommand{\ttower}[1]{\tower[\my@ttows]{#1}}
% bipartite tower of arbitrary depth
\newcommand{\btower}[1]{\tower{#1}}
\makeatother

\newcommand{\bitt}[2]{\btower{#1, #2}}
\newcommand{\bittt}[3]{\btower{#1, #2, #3}}
\newcommand{\bitttt}[4]{\btower{#1, #2, #3, #4}}

% tripartite 2-story tower
\newcommand{\tritt}[3]{%
  \ensuremath{\mathinner{%
      \begin{tabular}[c]{@{ }c@{ }}
        \strut\ensuremath{#1}\ \hfil\vrule\hfil\ \ensuremath{#2}\\
        \hline
        \strut\ensuremath{#3}\\
      \end{tabular}
  }}
}


%%%%% drawing %%%%%
\usepackage{tikz-qtree}

%%% tikz (for matrices)
\usepackage{tikz}
\usetikzlibrary{calc}


%%%%% borrowed shapes from other math libraries %%%%%

%%% corners, bigplus, harpoons from mathabx w/o loading the whole font
% corners
\DeclareFontFamily{U}{mathb}{}
\DeclareFontShape{U}{mathb}{m}{n}{%
      <5> <6> <7> <8> <9> <10> gen * mathb
      <10.95> mathb10 <12> <14.4> <17.28> <20.74> <24.88> mathb12 }{}
\DeclareSymbolFont{mathb}{U}{mathb}{m}{n}
\DeclareMathSymbol{\lcorners}{4}{mathb}{"76}% name to be checked
\DeclareMathSymbol{\rcorners}{5}{mathb}{"77}% name to be checked
\DeclareMathSymbol{\acorner}{4}{mathb}{"78}% name to be checked
\DeclareMathSymbol{\bcorner}{5}{mathb}{"79}% name to be checked
\DeclareMathSymbol{\ccorner}{4}{mathb}{"7A}% name to be checked
\DeclareMathSymbol{\dcorner}{5}{mathb}{"7B}% name to be checked
% bigplus
\DeclareFontFamily{U}{mathx}{\hyphenchar\font45}
\DeclareFontShape{U}{mathx}{m}{n}{%
    <5> <6> <7> <8> <9> <10>
    <10.95> <12> <14.4> <17.28> <20.74> <24.88>
    mathx10
}{}
\DeclareSymbolFont{mathx}{U}{mathx}{m}{n}
\DeclareMathSymbol{\bigplus}{1}{mathx}{"90}
% harpoons
\DeclareFontFamily{U}{matha}{}
\DeclareFontShape{U}{matha}{m}{n}{%
  <5> <6> <7> <8> <9> <10> gen * matha
  <10.95> matha10 <12> <14.4> <17.28> <20.74> <24.88> matha12 }{}
\DeclareSymbolFont{matha}{U}{matha}{m}{n}
\DeclareMathSymbol{\downupharpoons}{3}{matha}{"EB}

%%% powerset symbol from mnsymbol
\DeclareFontFamily{U}{MnSymbolC}{}
\DeclareSymbolFont{mnsymbols}{U}{MnSymbolC}{m}{n}
\DeclareFontShape{U}{MnSymbolC}{m}{n}{%
<-6> MnSymbolC5
<6-7> MnSymbolC6
<7-8> MnSymbolC7
<8-9> MnSymbolC8
<9-10> MnSymbolC9
<10-12> MnSymbolC10
<12-> MnSymbolC12}{}
\DeclareMathSymbol{\powerset}{\mathop}{mnsymbols}{180}

%%% right triangles from wasysym
\DeclareFontFamily{U}{wasy}{}
\DeclareFontShape{U}{wasy}{m}{n}{%
<-6> wasy5
<6-7> wasy6
<7-8> wasy7
<8-9> wasy8
<9-10> wasy9
<10-> wasy10}{}
\DeclareSymbolFont{wasy}{U}{wasy}{m}{n}
\DeclareMathSymbol{\RHD}{\mathbin}{wasy}{"11}
\let\rhd\undefined
\DeclareMathSymbol\rhd{\mathbin}{wasy}{"03}

\geometry{landscape, margin=0.5in}
\setlength{\parindent}{0pt}
\setlength{\mathindent}{0pt}
\delimitershortfall=-1pt
\setlist{leftmargin=*}

\newcommand{\M}{\text{\sffamily\bfseries M}}
\newcommand{\F}{\text{\sffamily\bfseries F}}

\begin{document}

\setlength\abovedisplayskip{0pt}
\setlength\belowdisplayskip{0pt}
\setlength\abovedisplayshortskip{0pt}
\setlength\belowdisplayshortskip{0pt}


\begin{minipage}[t]{0.365\textwidth} % Cont Applicators
\begin{spreadlines}{0pt}
\begin{align*}
  m \fsl n
  &\coloneq
  \begin{cases}
    m\,n
    &\textbf{if}\ m \dblcolon \alpha\sto\beta,\ n \dblcolon \alpha \\
    \l k \dt m\,\p{\l f \dt n\,\p{\l x \dt k\,\p{f \fsl x}}}
    &\textbf{otherwise}
  \end{cases} \\
  %
  m \bsl n
  &\coloneq
  \begin{cases}
    n\,m
    &\textbf{if}\ n \dblcolon \alpha\sto\beta,\ m \dblcolon \alpha \\
    \l k \dt m\,\p{\l x \dt n\,\p{\l f \dt k\,\p{x \bsl f}}}
    &\textbf{otherwise}
  \end{cases}
\end{align*}
\end{spreadlines}
\end{minipage}
%
%
\begin{minipage}[t]{0.35\textwidth}
  \begin{itemize}[label={}]
  \item
    The usual applicators for left and right continuized FA
  \item
    State.Set monad combinators for managing drefs
  \item
    A slightly tweaked lower operator to filter out falsity (more like
    traditional standard dynamic semantics). This is not essential, but makes
    the bookkeeping a little smoother.
\end{itemize}
\end{minipage}
%
%
\begin{minipage}[t]{0.285\textwidth} % State.Set Grammar
\begin{spreadlines}{0pt}
\begin{align*}
  &&
  \eta\,x &\coloneq
  \l g \dt \set{\pair{x}{g}} \\
  %
  &&
  m^{\bind} &\coloneq
  \l kg \dt \uset{k\,x\,g' \giv \pair{x}{g'} \in m\,g} \\
  %
  &&
  m^\dnar &\coloneq
  m\,\p{\l xs \dt \set{\pair{x}{s} \giv x \neq \cn{F}}} \\
  %
  &&
  x^\upar &\coloneq
  \p{\eta\,x}^{\bind} = \l k \dt k\,x \\
  %
  &&
  x^\reset &\coloneq
  \p{x^\dnar}^{\bind}
\end{align*}
\end{spreadlines}
\end{minipage}


\dotbreak

\begin{minipage}[t]{0.5\textwidth} % Basic Definite Fragment
\begin{spreadlines}{0pt}
\begin{align*}
  \textbf{circle} &\coloneq
  \cn{circ} \\
  %
  \textbf{square} &\coloneq
  \cn{sq} \\
  %
  \textbf{in} &\coloneq
  \cn{in} \\
  %
  \textbf{the}_u &\coloneq
    \l ckg \col |G_u| = 1 \dt G, \\
  &\hphantom{{}\coloneq \l ckg \col}
    \text{where}\ 
    G = \uset{k\,x\,g' \giv \pair{\cn{T}}{g'} \in c\,x\,g^{u\mapsto x}} \\
\end{align*}  
\end{spreadlines}
\end{minipage}
%
%
\begin{minipage}[t]{0.5\textwidth} % Basic Definite Fragment Notes
\begin{spreadlines}{0pt}
\begin{itemize}
  \item
    $\textbf{the}_u$ is just $\textbf{a}_u$ plus a uniqueness presupposition.
    The presupposition restricts what the \emph{set} of outputs is allowed to
    look like: they need to all agree on the value of $u$
  \item
    That means its uniqueness effect is \emph{delayed} until some program
    containing it is evaluated, (i.e., until its continuation is delimited).
    This sort of delayed global test on the set of outputs is very much like a
    postsupposition (Brasoveanu 2012, Henderson 2014), but here it is
    regulated by continuations, rather than logical subscripts (Charlow 2014)
\end{itemize}
\end{spreadlines}
\end{minipage}

\dotbreak

\begin{minipage}[t]{0.5\textwidth} % Basic Nested Definite Derivs
\begin{align*} % Resetting a description
  \sv{\textrm{the square}}
  %
  &\leadsto
  \p{%
    \bitt{\textbf{the}_v\,\p{\l y \dt \hole}}{y}
    %
    \BSL{%
    %
    \bitt{\hole}{\textbf{square}}
    }
  }^{\dnar\,\reset}
  %
  \leadsto
  \p{%
    \textbf{the}_v\,\p{\l yg \dt \set{\pair{\cn{sq}\,y}{g}}}
  }^\reset \\
  %
  %
  &\leadsto
  \p{%
    \bitt{\l g \dt \uset{\hole\,g^{v\mapsto y} \giv \cn{sq}\,y}}{y}
  }^\reset
  %
  \leadsto
  \bitt{%
    \l g \col |G_v| = 1 \dt \uset{\hole\,g^{v\mapsto y} \giv \cn{sq}\,y}
  }{y}
\end{align*}
%
\vspace{3em}
%
\begin{align*} % Absolute Definite Deriv
  &
  \sv{\textrm{the circle in the square}} \\
  %
  %
  &
  \p{%
    \bitt{\textbf{the}_u\,\p{\l x \dt \hole}}{x}
    %
    \BSL{%
    %
    \p{%
      \bitt{\hole}{\textbf{circle}}
      %
      \MSL{%
      %
      \bitt{\hole}{\textbf{in}}
      %
      \FSL{%
      %
      \p{%
        \bitt{\textbf{the}_v\,\p{\l y \dt \hole}}{y}
        %
        \BSL{%
        %
        \bitt{\hole}{\textbf{square}}
        }
      }^{\dnar\,\reset} } }
    } }
  }^{\dnar\,\reset} \\
  %
  %
  &
  \p{%
    \bitt{\textbf{the}_u\,\p{\l x \dt \hole}}{x}
    %
    \BSL{%
    %
    \p{%
      \bitt{\hole}{\textbf{circle}}
      %
      \MSL{%
      %
      \bitt{\hole}{\textbf{in}}
      %
      \FSL{%
      %
      \bitt{%
        \l g \col |G_v| = 1 \dt \uset{\hole\,g^{v\mapsto y} \giv \cn{sq}\,y}
      }{y}
      } }
    } }
  }^{\dnar\,\reset} \\
  %
  %
  &
  \p{%
    \bitt{%
      \textbf{the}_u\,\p{%
        \l xg \col |G_v| = 1 \dt \uset{\hole\,g^{v\mapsto y} \giv \cn{sq}\,y}
      }
    }{%
      \cn{circ}\,x \land \cn{in}\,y\,x
    }
  }^{\dnar\,\reset} \\
  %
  %
  &
  \p{%
    \bitt{%
      \l g \col |G_v| = 1 \dt
        \uset{%
          \hole\,g^{\substack{u\mapsto x\\ v\mapsto y}}
        \giv
          \cn{sq}\,y,\ \cn{circ}\,x,\ \cn{in}\,y\,x
        }
    }{x}
  }^\reset \\
  %
  %
  &
  \bitt{%
    \l g \col |G'_u| = |G_v| = 1 \dt
      \uset{%
        \hole\,g^{\substack{u\mapsto x\\ v\mapsto y}}
      \giv
        \cn{sq}\,y,\ \cn{circ}\,x,\ \cn{in}\,y\,x
      }
  }{x}
\end{align*}
\end{minipage}
%
%
\begin{minipage}[t]{0.5\textwidth} % Absolute Definite Notes
  \begin{itemize}
    \item
      Note that resetting $\sv{\textrm{the square}}$ in the last reduction
      step here has no effect on its semantic shape, because it's essentially
      $\sv{\textrm{a square}}$.
    \item
      But it does fix the presupposition; For any input $g$, $G$ will be equal
      to $\set{\pair{y}{g^{v\mapsto y}} \giv \cn{sq}\,y}$, and the presup will
      require all those $g$'s to map $v$ to the same square (which will only
      be possible if there's exactly one square available to assign $v$ to
      in the first place).
  \end{itemize} 
  %
  \vspace{2em}
  %
  \textbf{Absolute Reading}
  %
  \begin{itemize}
    \item
      The inner definite is reset, freezing its presupposition as above. When
      the input assignment $g$ is eventually inserted, we will have $G =
      \set{\pair{y}{g^{v\mapsto y}} \giv \cn{sq}\,y}$, and the presupposition
      will guarantee that $g'\,v$ is constant across the outputs.
    \item
      As with the inner DP, the host DP's presupposition is fixed when it is
      reset. This time, we have $G' = \set{\pair{x}{g^{\substack{u\mapsto x\\
      v\mapsto y}}} \giv \cn{sq}\,y,\ \cn{circ}\,x,\ \cn{in}\,y\,x}$, where
      $g$ is whatever the input happens to be. In particular, all the outputs
      will now need to agree on the value of $u$ in addition to $v$, which
      will only be possible if there's exactly one circle in the square that
      all outputs assign to $v$.
  \end{itemize}
\end{minipage}

\newpage

\begin{minipage}[t]{0.5\textwidth} % Basic Definite Derivs Cont'd
\begin{align*} % Relative Reading Deriv
  &
  \sv{\textrm{the circle in the square}} \\
  %
  %
  &
  \p{%
    \bitt{\textbf{the}_u\,\p{\l x \dt \hole}}{x}
    %
    \BSL{%
    %
    \p{%
      \bitt{\hole}{\textbf{circle}}
      %
      \MSL{%
      %
      \bitt{\hole}{\textbf{in}}
      %
      \FSL{%
      %
      \p{%
        \bitt{\textbf{the}_v\,\p{\l y \dt \hole}}{y}
        %
        \BSL{%
        %
        \bitt{\hole}{\textbf{square}}
        }
      }^{\dnar} } }
    } }
  }^{\dnar\,\reset} \\
  %
  %
  &
  \p{%
    \bitt{\textbf{the}_u\,\p{\l x \dt \hole}}{x}
    %
    \BSL{%
    %
    \p{%
      \bitt{\hole}{\textbf{circle}}
      %
      \MSL{%
      %
      \bitt{\hole}{\textbf{in}}
      %
      \FSL{%
      %
      \bitt{%
        \l g \dt \uset{\hole\,g^{v\mapsto y} \giv \cn{sq}\,y}
      }{y}
      } }
    } }
  }^{\dnar\,\reset} \\
  %
  %
  &
  \p{%
    \bitt{%
      \textbf{the}_u\,\p{%
        \l xg \dt \uset{\hole\,g^{v\mapsto y} \giv \cn{sq}\,y}
      }
    }{%
      \cn{circ}\,x \land \cn{in}\,y\,x
    }
  }^{\dnar\,\reset} \\
  %
  %
  &
  \p{%
    \bitt{%
      \l g \dt
        \uset{%
          \hole\,g^{\substack{u\mapsto x\\ v\mapsto y}}
        \giv
          \cn{sq}\,y,\ \cn{circ}\,x,\ \cn{in}\,y\,x
        }
    }{x}
  }^{\reset} \\
  %
  %
  &
  \bitt{%
    \l g \col |G_u| = |G_v| = 1 \dt
      \uset{%
        \hole\,g^{\substack{u\mapsto x\\ v\mapsto y}}
      \giv
        \cn{sq}\,y,\ \cn{circ}\,x,\ \cn{in}\,y\,x
      }
  }{x}
\end{align*}
\end{minipage}
%
%
\begin{minipage}[t]{0.5\textwidth} % Relative Definite Comments
  \textbf{Relative Reading} (cf.\ Haddock, Champollion and Saurland)

  \begin{itemize}
    \item
      The only difference here is that we do not reset the inner DP, which
      staves off its presupposition until more information is accumulated in
      its scope
    \item
      But now when the outer DP is reset, it sets the presuppositions
      of \emph{both} definites
    \item
      For any input $g$, $G =
      \set{\pair{x}{g^{\substack{u\mapsto x\\ v\mapsto y}}} \giv \cn{sq}\,x,\
        \cn{circ}\,y,\ \cn{in}\,y\,x}$ is the set out outputs that map $u$
        onto a circle in some square that it maps to $v$.
    \item
      So requiring that there be exactly one such $v$ is tantamount to
      requiring that there be exactly one square \emph{that has a circle in
      it} and exactly one circle \emph{in that square}. In other words, there
      should be exactly one pair $\pair{x}{y}$ in $\cn{circ} \times \cn{sq}$
      such that $\cn{in}\,y\,x$.
  \end{itemize}
\end{minipage}

\dotbreak

\begin{minipage}[t]{0.5\textwidth} % Plural Fragment
\begin{spreadlines}{0pt}
\begin{align*}
  \M_u &\coloneq
  \l G \dt
    \set{%
      \pair{\cdot}{g} \in G
    \giv
      \neg\exists\pair{\cdot}{g'} \in G \dt g'\,u \sqsupset g\,u
    } \\
  %
  \textbf{the}_u &\coloneq
    \l ckg \col |G_u| = 1 \dt G, \\
  &\hphantom{{}\coloneq{}}
    \text{where}\ 
    G = \M_u\uset{k\,x\,g' \giv \pair{\cn{T}}{g'} \in c\,x\,g^{u\mapsto x}} \\
  %
  \textbf{-s} &\coloneq
  \l Px \dt x \in \set{\mathlarger{\oplus} P' \giv P' \subseteq P}
\end{align*}  
\end{spreadlines}
\end{minipage}
%
%
\begin{minipage}[t]{0.5\textwidth} % Plural Fragment Notes
\begin{spreadlines}{0pt}
  \begin{itemize}
    \item
      $\textbf{-s}$ is a boilerplate plural morpheme that builds sums from the
      atoms in its complement.
    \item
      $\M_u$ is a kind of maximization operator on outputs (Brasoveanu
      2012, Charlow 2014). It filters out those assignments in $g \in G$ that
      are strictly dominated, in the sense that they assign $u$ to a value
      that is a proper part of something assigned to $u$ by some other $g' \in
      G$.
  \end{itemize}
\end{spreadlines}
\end{minipage}

\dotbreak

Derivations \dots


\newpage

\dotbreak

\begin{minipage}[t]{0.5\textwidth} % DP Superlative Fragment
\begin{spreadlines}{0pt}
\begin{align*}
  \M_u^f &\coloneq
  \l G \dt
    \set{%
      \pair{\cdot}{g} \in G
    \giv
      \neg\exists\pair{\cdot}{g'} \in G \dt f\,(g'\,u)\,(g\,u)
    } \\
  %
  \textbf{older} &\coloneq
    \l xy \dt \cn{age}\,x > \cn{age}y \\
  %
  \textbf{est}_u &\coloneq
  \l f \dt \M_u^f \\
  %
  \textbf{oldest}_u &=
  \textbf{est}\,\textbf{older} =
  \l G \dt
    \set{%
      \pair{\cdot}{g} \in G
    \giv
      \neg\exists\pair{\cdot}{g'} \in G \dt \cn{age}\,(g'\,u) > \cn{age}\,(g\,u)
    } \\
  \textbf{the}_u &\coloneq
  \l \mathcal{M}ckg \dt |G_u| = 1 \dt G, \\
  &\hphantom{{}\coloneq{}}
    \text{where}\ 
    G = \mathcal{M}_u\uset{%
      k\,x\,g'
    \giv
      \pair{\cn{T}}{g'} \in c\,x\,g^{u\mapsto x}
    }
\end{align*}
\end{spreadlines}
\end{minipage}
%
%
\begin{minipage}[t]{0.5\textwidth} % DP Superlative Notes
\begin{itemize}
  \item
    $\textbf{est}$ abstracts over the ordering function that $\M$ uses to
    compare individuals. In the case of pluralities, $\M_u$ filters away any
    output that assigns $u$ to a \emph{smaller sum} than it could have (ie., a
    smaller sum than one of the other outputs assigns to $u$). In the case of
    $\textbf{oldest}$, $\M_u$ filters outputs that assign $u$ to a
    \emph{younger} individual than they could have.
  \item
    And now here's the swim move: $\textbf{the}$ absorbs the max operator and
    then carries it along for the ride. ``Absolute'' definite become absolute
    superlatives; ``Haddock'' definites become relative superlatives. This
    part is my favorite.
\end{itemize}
\end{minipage}

\dotbreak

\begin{minipage}[t]{0.5\textwidth} % DP Superlative Derivs
\begin{align*} % Resetting a superlative
  &
  \sv{\textrm{the oldest squirrel}} \\
  %
  %
  &\leadsto
  \p{%
    \bitt{\textbf{the}_v\,\M_v^\cn{ag}\,\p{\l y \dt \hole}}{y}
    %
    \BSL{%
    %
    \bitt{\hole}{\textbf{squirrel}}
    }
  }^{\dnar\,\reset}
  %
  \leadsto
  \p{%
    \textbf{the}_v\,\M_v^\cn{ag}\,\p{\l yg \dt \set{\pair{\cn{sq}\,y}{g}}}
  }^\reset \\
  %
  %
  &\leadsto
  \p{%
    \bitt{\l g \dt \M_v^\cn{ag}\uset{\hole\,g^{v\mapsto y} \giv \cn{sq}\,y}}{y}
  }^\reset
  %
  \leadsto
  \bitt{%
    \l g \col |G_v| = 1 \dt
      \uset{%
        \hole\,g^{v\mapsto y}
      \giv
        \begin{array}{L}
          \cn{sq}\,y,\\ \forall z\col\cn{sq} \dt \neg\,\cn{older}\,z\,y
        \end{array}
      }
  }{y}
\end{align*}
%
\vspace{3em}
%
\begin{align*} % Absolute Superlative Deriv
  &
  \sv{\textrm{the circus with the oldest squirrel}} \\
  %
  %
  &
  \p{%
    \bitt{\textbf{the}_u\,\p{\l x \dt \hole}}{x}
    %
    \BSL{%
    %
    \p{%
      \bitt{\hole}{\textbf{circus}}
      %
      \MSL{%
      %
      \bitt{\hole}{\textbf{with}}
      %
      \FSL{%
      %
      \p{%
        \bitt{\textbf{the}_v\,\M_v^\cn{ag}\,\p{\l y \dt \hole}}{y}
        %
        \BSL{%
        %
        \bitt{\hole}{\textbf{squirrel}}
        }
      }^{\dnar\,\reset} } }
    } }
  }^{\dnar\,\reset} \\
  %
  %
  &
  \p{%
    \bitt{\textbf{the}_u\,\p{\l x \dt \hole}}{x}
    %
    \BSL{%
    %
    \p{%
      \bitt{\hole}{\textbf{circus}}
      %
      \MSL{%
      %
      \bitt{\hole}{\textbf{with}}
      %
      \FSL{%
      %
      \bitt{%
        \l g \col |G_v| = 1 \dt
          \uset{%
            \hole\,g^{v\mapsto y}
          \giv
            \begin{array}{L}
              \cn{sq}\,y,\\ \forall z\col\cn{sq} \dt \neg\,\cn{older}\,z\,y
            \end{array}
          }
      }{y}
      } }
    } }
  }^{\dnar\,\reset} \\
  %
  %
  &
  \p{%
    \bitt{%
      \textbf{the}_u\,\p{%
        \l xg \col |G_v| = 1 \dt
          \uset{%
            \hole\,g^{v\mapsto y}
          \giv
            \cn{sq}\,y,\
            \forall z\col\cn{sq} \dt \neg\,\cn{older}\,z\,y
          }
      }
    }{%
      \cn{circ}\,x \land \cn{with}\,y\,x
    }
  }^{\dnar\,\reset} \\
  %
  %
  &
  \p{%
    \bitt{%
      \l g \col |G_v| = 1 \dt
        \uset{%
          \hole\,g^{\substack{u\mapsto x\\ v\mapsto y}}
        \giv
          \begin{array}{L}
          \cn{sq}\,y,\ \cn{circ}\,x,\ \cn{with}\,y\,x, \\
          \forall z\col\cn{sq} \dt \neg\,\cn{older}\,z\,y
          \end{array}
        }
    }{x}
  }^\reset \\
  %
  %
  &
  \bitt{%
    \l g \col |G'_u| = |G_v| = 1 \dt
      \uset{%
        \hole\,g^{\substack{u\mapsto x\\ v\mapsto y}}
      \giv
        \begin{array}{L}
        \cn{sq}\,y,\ \cn{circ}\,x,\ \cn{with}\,y\,x, \\
        \forall z\col\cn{sq} \dt \neg\,\cn{older}\,z\,y
        \end{array}
      }
  }{x}
\end{align*}
\end{minipage}
%
%
\begin{minipage}[t]{0.5\textwidth} % DP Superlative Notes
  \begin{itemize}
    \item
      Resetting the superlative DP has two effects: (1) it fixes the set of
      individuals that the superlative operator $\M$ compares, in this case
      squirrels with respect to age; and (2) it freezes the presupposition
      associated with the definite article.
    \item
      Here $G = \set{\pair{y}{g^{v\mapsto y}} \giv \cn{sq}\,y,\ \forall
      z\col\cn{sq} \dt \neg\,\cn{older}\,z\,y}$. This set could in principle
      have multiple winners (multiple squirrels with the same age). The definite
      determiner rules that out; all the $g \in G$ need to point $v$ to
      \emph{the same} squirrel.
  \end{itemize}
  %
  \vspace{4em}
  \textbf{Absolute Superlative Reading}
  %
  \begin{itemize}
    \item
      When the inner DP is reset, it has the effect just demonstrated: the set
      of outputs is restricted to those that map $v$ to the (unique) oldest
      squirrel.
    \item
      The outer definite has no explicit maximization operator. For
      simplicity, I assume that in such cases, it defaults to the basic plural
      max operator introduced above. This will have no effect when the
      predicates are all singular, but the uniqueness presupposition it brings
      to bear is still forceful.
    \item
      Altogether, after the host DP has been reset, we will have an
      essentially nondeterministic (continuized) individual with two
      presuppositions: (1) there is exactly one squirrel with no elders; and (2)
      there is exactly one circus that he is with.
  \end{itemize}
\end{minipage}

\newpage

\begin{minipage}[t]{0.5\textwidth} % DP Superlative Derivs Cont'd
\begin{align*} % Relative Superlative Deriv
  &
  \sv{\textrm{the circus with the oldest squirrel}} \\
  %
  %
  &
  \p{%
    \bitt{\textbf{the}_u\,\p{\l x \dt \hole}}{x}
    %
    \BSL{%
    %
    \p{%
      \bitt{\hole}{\textbf{circus}}
      %
      \MSL{%
      %
      \bitt{\hole}{\textbf{with}}
      %
      \FSL{%
      %
      \p{%
        \bitt{\textbf{the}_v\,\M_v^\cn{ag}\,\p{\l y \dt \hole}}{y}
        %
        \BSL{%
        %
        \bitt{\hole}{\textbf{squirrel}}
        }
      }^{\dnar} } }
    } }
  }^{\dnar\,\reset} \\
  %
  %
  &
  \p{%
    \bitt{\textbf{the}_u\,\p{\l x \dt \hole}}{x}
    %
    \BSL{%
    %
    \p{%
      \bitt{\hole}{\textbf{circus}}
      %
      \MSL{%
      %
      \bitt{\hole}{\textbf{with}}
      %
      \FSL{%
      %
      \bitt{%
        \l g \dt \M_v^\cn{ag}\uset{\hole\,g^{v\mapsto y} \giv \cn{sq}\,y}
      }{y}
      } }
    } }
  }^{\dnar\,\reset} \\
  %
  %
  &
  \p{%
    \bitt{%
      \textbf{the}_u\,\p{%
        \l xg \dt \M_v^\cn{ag}\uset{\hole\,g^{v\mapsto y} \giv \cn{sq}\,y}
      }
    }{%
      \cn{circ}\,x \land \cn{with}\,y\,x
    }
  }^{\dnar\,\reset} \\
  %
  %
  &
  \p{%
    \bitt{%
      \l g \dt
        \uset{%
          \hole\,g^{\substack{u\mapsto x\\ v\mapsto y}}
        \giv
          \cn{sq}\,y,\ \cn{circ}\,x,\ \cn{with}\,y\,x
        }
    }{x}
  }^{\reset} \\
  %
  %
  &
  \bitt{%
    \l g \col |G_u| = |G_v| = 1 \dt
      \uset{%
        \hole\,g^{\substack{u\mapsto x\\ v\mapsto y}}
      \giv
        \cn{sq}\,y,\ \cn{circ}\,x,\ \cn{with}\,y\,x
      }
  }{x}
\end{align*}
\end{minipage}
%
%
\begin{minipage}[t]{0.5\textwidth} % Relative Definite Comments
  \textbf{Relative Reading} (cf.\ Haddock, Champollion and Saurland)

  \begin{itemize}
    \item
      The only difference here is that we do not reset the inner DP, which
      staves off its presupposition until more information is accumulated in
      its scope
    \item
      But now when the outer DP is reset, it sets the presuppositions
      of \emph{both} definites
    \item
      For any input $g$, $G =
      \set{\pair{x}{g^{\substack{u\mapsto x\\ v\mapsto y}}} \giv \cn{sq}\,x,\
        \cn{circ}\,y,\ \cn{in}\,y\,x}$ is the set out outputs that map $u$
        onto a circle in some square that it maps to $v$.
    \item
      So requiring that there be exactly one such $v$ is tantamount to
      requiring that there be exactly one square \emph{that has a circle in
      it} and exactly one circle \emph{in that square}. In other words, there
      should be exactly one pair $\pair{x}{y}$ in $\cn{circ} \times \cn{sq}$
      such that $\cn{in}\,y\,x$.
  \end{itemize}
\end{minipage}


\dotbreak

\begin{minipage}[t]{0.6\textwidth} % Adding Focus
\begin{spreadlines}{0em}
\begin{align*}
  % \M_u^f &=
  % \lambda G \dt
  %   \set{%
  %     \pair{\cn{\cdot}}{g} \in G
  %   \giv
  %     \neg\exists\pair{\cdot}{g'} \in G \dt f\,(g\,u)\,(g'\,u)
  %   } \\
  % %
  \M_u^f &=
  \lambda G \dt
    \set{%
      \pair{\cn{\alpha}}{g}
    \giv
      \pair{\pair{\cn{\alpha}}{\cdot}}{g} \in G
      ,\ \, 
      \cn{truthy}\,\alpha
      ,\ \, 
      \neg\exists\pair{\pair{\cdot}{\beta}}{g'} \in G \dt
        \mathsmaller{\bigvee}\!\beta \land f\,(g\,u)\,(g'\,u)
    } \\
  %
  \textbf{larger} &=
  \lambda xy \dt \cn{size}\,x < \cn{size}\,y \\
  %
  \textbf{est}_u &=
  \lambda f \dt \M_u^f\\
  %
  \textbf{largest}_u &=
  \textbf{est}_u\,\textbf{larger} =
  \M_u^\cn{sz} =
  \lambda G \dt
    \set{%
      \pair{\cn{T}}{g}
    \giv
      \pair{\cn{T}}{g} \in G
      \land
      \neg\exists\pair{\cn{T}}{g'} \in G \dt
        \cn{size}\,(g\,u) < \cn{size}\,(g'\,u)
    } \\
  %
  \textbf{the}_u &=
  \lambda\mathcal{M}ckg \col |G'_u| = 1 \dt G',
  \ \textrm{where} \ 
  G' = \mathcal{M}\uset{%
    k\,x\,g'
  \giv
    x\in \mathcal{D}_e,\ \pair{\cn{T}}{g'} \in c\,x\,g^{u\mapsto x}
  }
\end{align*}
\end{spreadlines}
\end{minipage}
%
%
\begin{minipage}[t]{0.4\textwidth} % Focus Notes
Adding focus-sensitivity

\end{minipage}

\dotbreak

\begin{minipage}[t]{0.6\textwidth} % Focus Derivs
\begin{align*}
  \bitttt{\hole}{%
    \cn{j}^\F \star \p{\l j \dt \hole}
  }{%
    j^\rhd \star \p{\l x \dt \hole}
  }{x}
%
\FSL{%
  \bitttt{\hole}{\hole}{\hole}{\cn{drew}}
}
%
\BSL{%
  \bitttt{%
    \lambda g \dt
      \M_u \uset{%
        \hole\,g^{u\mapsto y}
      \giv
        \cn{sq}\,y
      }
  }{\hole}{\hole}{y}
}
\end{align*}  
\end{minipage}
%
%
\begin{minipage}[t]{0.4\textwidth}
John drew the largest square
\end{minipage}

\dotbreak

\begin{minipage}[t]{0.5\textwidth} % Focus-State Monad Fragment
\begin{spreadlines}{0pt}
\begin{align*}
  \mathcal{F}\alpha
  &\coloneq
  \sigma \sto \typepair{%
    \set{\typepair{\alpha}{\sigma}}
  }{%
    \set{\typepair{\alpha}{\sigma}}
  } \\
  %
  \eta\,x
  &\coloneq
  \l g \dt \pair{\set{\pair{x}{g}}}{\set{\pair{x}{g}}} \\
  %
  m \bind f
  &\coloneq
  \l g \dt
    \thickpair{%
      \uset{\p{f\,x\,g'}_1 \giv \pair{x}{g'} \in \p{m\,g}_1}
    }{%
      \uset{\p{f\,x\,g'}_2 \giv \pair{x}{g'} \in \p{m\,g}_2}
    }
\end{align*}
\end{spreadlines}
\end{minipage}
%
%
\begin{minipage}[t]{0.5\textwidth} % Focus-State Monad Notes
\begin{itemize}
  \item
    Rolling focus and state simultaneously to grease the wheels
\end{itemize}
\end{minipage}

\dotbreak

\begin{minipage}[t]{0.5\textwidth} % Only Fragment
% \begin{spreadlines}{0pt}
\begin{align*}
  \cn{true} &\coloneq
  \l G \dt \bigvee\set{\alpha \giv \pair{\alpha}{g} \in G} \\
  %
  \textbf{not} &\coloneq
  \l mg \dt \eta\,\p{\neg \p{m\,g}}\,g \\
  %
  \textbf{only}_u &\coloneq 
  \l G \col \cn{true}\,G \dt \M_u\,G \\
  %
  \textbf{the}_u &\coloneq
  \l \mathcal{M}ckg \dt |G_u| = 1 \dt G, \\
  &\hphantom{{}\coloneq{}}
    \text{where}\ 
    G = \mathcal{M}_u\uset{%
      k\,x\,g'
    \giv
      \pair{\cn{T}}{g'} \in c\,x\,g^{u\mapsto x}
    }
\end{align*}
% \end{spreadlines}
\end{minipage}
%
%
\begin{minipage}[t]{0.5\textwidth} % Only Fragment Notes
\begin{spreadlines}{0pt}
\begin{itemize}
  \item
    $\cn{true}$ and $\textbf{not}$ are bog standard dynamic booleans: an
    update is ``true'' iff it generates at least one successful output
  \item
    Adjectival $\textbf{only}$ is just $\M$ plus a presupposition (cf.\
    Coppock and Beaver 2012, 2014)!
  \item
    So here's the swim move: feed $\textbf{only}$ into $\textbf{the}$, and
    then let the scope of the exclusivity ride the scope of the determiner.
    This will predict relative readings for adj $\textbf{only}$.
\end{itemize}
\end{spreadlines}
\end{minipage}

\dotbreak

John sold the only cars \\

Anna didn't give the only good talk



\end{document}
