
%%%%% deal with the excessive number of packages %%%%%
\usepackage{etex}

\usepackage{silence} % suppres font shape warnings
  \WarningFilter{latexfont}{Font shape}
\usepackage{ifthen} % for conditional macros

%%%%% general math %%%%%
\usepackage[tbtags]{mathtools} % loads amsmath
\usepackage{amssymb,stmaryrd}
  % \SetSymbolFont{stmry}{bold}{U}{stmry}{m}{n}
\usepackage{relsize} % change size of math operators
\usepackage{exscale} % change size of math operators arbitrarily
\usepackage{scalerel} % change size of math delimiters
% prevent align env at top of minipage from adding additional padding
\usepackage{etoolbox}
  \makeatletter
  \pretocmd\start@align{%
    \if@minipage\kern-\topskip\kern-\abovedisplayskip\fi
  }{}{}
  \makeatother

%%% all-purpose math macros
% parens
\DeclarePairedDelimiterX\PARENS[1](){#1}
\newcommand{\p}[1]{\PARENS*{#1}}
% set comprehension: \set{ ... \giv ... } = { ... | ... }
\providecommand{\giv}{}
\DeclarePairedDelimiterX\SET[1]\{\}{%
  \renewcommand{\giv}{\nonscript\:\delimsize\vert\nonscript\:\mathopen{}}
  #1
}
\newcommand{\set}[1]{\SET*{#1}}
% grand union: \uset{ ... \giv ... } = U{ ... | ... }
\DeclarePairedDelimiterXPP\USET[1]{\!\bigcup}\{\}{}{%
  \renewcommand{\giv}{\nonscript\:\delimsize\vert\nonscript\:\mathopen{}}
  #1
}
\newcommand{\uset}[1]{\USET*{#1}}
% tuples
\DeclarePairedDelimiterX\TUP[1]\langle\rangle{#1}
% tuple of arbitrary length
\makeatletter
\newcommand{\tup}[1]{%
  \ensuremath{%
    \TUP*{\my@tups #1,\relax\noexpand\@eolst}%
  }%
}
\def\my@tups #1,#2\@eolst{%
  \ifx\relax#2\relax
    #1%
  \else
    #1,\pt \my@tups #2\@eolst%
  \fi
}
\makeatother
\newcommand{\pair}[2]{\tup{#1, #2}}
\newcommand{\thickpair}[2]{\TUP*{#1,\ \ #2}}
\newcommand{\typepair}[2]{#1 \ast #2}
% verts
\DeclarePairedDelimiterX\ABS[1]||{#1}
\newcommand{\abs}[1]{\ABS*{#1}}

%%%%% examples, lists, footnotes, citations %%%%%
\usepackage[bottom,multiple]{footmisc} % force footnotes to bottom of page
\usepackage{enumitem} % customize lists
\usepackage{epltxfn} % expex examples in footnotes
\usepackage{expex} % for example sentences
\newcommand{\excite}[3][]{% citation at right edge of example
  \ifthenelse{\equal{#1}{}}{% no opt arg (no prefix) 
    \ifthenelse{\equal{#3}{ibid}}{% no reference to look up
      \rightcomment{[\textit{ibid}:~#2]}
    }{% look up reference
      \rightcomment{[\citealt[#2]{#3}]}
    }
  }{% optional argument present
    \ifthenelse{\equal{#1}{citealias}}{% use citealias
      \rightcomment{[\citetalias{#3}:~#2]}
    }{% set opt arg as citation prefix
      \ifthenelse{\equal{#3}{ibid}}{% no reference to look up
        \rightcomment{[#1 \textit{ibid}:~#2]}
      }{% look up reference
        \rightcomment{[#1 \citealt[#2]{#3}]}
      }
    }
  }
}
% inline citation style abbreviations
\usepackage{natbib}
  \bibpunct[: ]{(}{)}{,}{a}{}{,}
  \newcommand{\BIBand}{\&}
  \setlength{\bibsep}{0pt}
  \setlength{\bibhang}{0.25in}
  \bibliographystyle{sp}
  \newcommand{\posscitet}[1]{\citeauthor{#1}'s (\citeyear{#1})}
  \newcommand{\possciteauthor}[1]{\citeauthor{#1}'s}
  \newcommand{\pgposscitet}[2]{\citeauthor{#1}'s (\citeyear{#1}:~#2)}
  \newcommand{\secposscitet}[2]{\citeauthor{#1}'s (\citeyear{#1}:~$\S$#2)}
  \newcommand{\pgcitealt}[2]{\citealt{#1}:~#2}
  \newcommand{\seccitealt}[2]{\citealt{#1}:~$\S$#2}
  \newcommand{\pgcitep}[2]{(\citealt{#1}:~#2)}
  \newcommand{\seccitep}[2]{(\citealt{#1}:~$\S$#2)}
  \newcommand{\pgcitet}[2]{\citeauthor{#1} (\citeyear{#1}:~#2)}
  \newcommand{\seccitet}[2]{\citeauthor{#1} (\citeyear{#1}:~$\S$#2)}



%%%%% tables %%%%%
\usepackage{array} % more control over column styles
\usepackage{booktabs} % better rules

%%% array columns without spaces
\newcolumntype{L}{@{}l@{}}
\newcolumntype{R}{@{}r@{}}
\newcolumntype{C}{@{}c@{}}



%%%%% layout and formatting %%%%%
\usepackage[margin=1in]{geometry} % margins
\usepackage{sectsty} % customize section title appearance
  \allsectionsfont{\normalsize}

%%% customize the abstract
\newcommand{\frontmatterspacing}[1]{%
  \small
  \topsep 10pt
  \advance\topsep by 3.5ex plus -1ex minus -0.2ex
  \setlength{\listparindent}{0em}
  \setlength{\itemindent}{0em}
  \setlength{\leftmargin}{#1}
  \setlength{\rightmargin}{\leftmargin}
  \setlength{\parskip}{0em}
}
\renewenvironment{abstract}{%
  \list{}{\frontmatterspacing{0.25in}}%
  \item\relax\textbf{\abstractname}
}{\endlist}

%%% font preferences
\usepackage[T1]{fontenc}
\usepackage[utf8]{inputenc}
\usepackage{libertine}
\usepackage[libertine,liby,bigdelims]{newtxmath}

%%% links
\RequirePackage[usenames]{xcolor}
\definecolor{splinkcolor}{rgb}{.0,.2,.4}
\RequirePackage[colorlinks,breaklinks,
                linkcolor=splinkcolor, 
                urlcolor=splinkcolor, 
                citecolor=splinkcolor,
                filecolor=splinkcolor,
                plainpages=false,
                pdfpagelabels,
                bookmarks=false,
                pdfstartview=FitH]{hyperref}
\newcommand{\doi}[1]{\url{http://dx.doi.org/#1}}
\urlstyle{rm}
\usepackage{hyperref}


%%%%% additional symbols and macros %%%%%%

%%% sp macros
\newcommand{\sv}[1]{\llbracket #1 \rrbracket}
\newcommand{\with}{\mathbin{\&}}

%%% basic convenience macros
\newcommand{\pt}{\hspace{1pt}}
\newcommand{\ppt}{\hspace{2pt}}
\newcommand{\col}{\pt{:}\ppt}
\newcommand{\dt}{\pt{.}\ppt}
\newcommand{\cn}[1]{{\sf #1}}
\newcommand{\cat}{\hspace{-2pt}\cdot\hspace{-2pt}}
\newcommand{\sto}{\mathbin{\shortrightarrow}}
\newcommand{\must}{\mathlarger{\mathlarger{\Box}}}
\newcommand{\rest}[2]{#1 \col #2}
\newcommand{\objl}[1]{`#1'}
\newcommand{\tru}{\textsf{\bfseries T}}
\newcommand{\fals}{\textsf{\bfseries F}}
\newcommand{\ms}[1]{\mathsmaller{#1}}

% phantom to width of second argument
\newcommand*\phantomas[3][c]{%
   \ifmmode
     \makebox[\widthof{$#2$}][#1]{$#3$}%
   \else
     \makebox[\widthof{#2}][#1]{#3}%
   \fi
}
% dots to break sections in handouts, etc
\newcommand{\dotbreak}[1][]{%
  % \vspace{0.5em}\dotfill\hspace{0.5\textwidth} \\ \textsc{#1}\vspace{0.5em}
  \ifthenelse{\equal{#1}{}}{%
    \vspace{0.5em}\dotfill\vspace{0.5em}
  }{%
    \vspace{0.5em}\dotfill\hspace{0.5\textwidth} \\ \textsc{#1}\vspace{0.5em}
  }
}
% two-column minipage, with widths controlled by optional param
\newenvironment{minisplit}[1][0.5]{%
  \newcommand{\splitmini}{%
    \end{minipage}%
    \begin{minipage}[t]{\dimexpr\textwidth-#1\textwidth\relax}%
  }
  \noindent\ignorespaces%
  \begin{minipage}[t]{#1\textwidth}%
}{%
  \end{minipage}%
  \ignorespacesafterend%
}

%%% judgment diacritics
\newcommand{\att}{${}^{\boldsymbol\gamma}$}
\newcommand{\yes}{\textsuperscript{\checkmark}}
\newcommand{\bad}{\textsuperscript{\small\#}}
\newcommand{\mar}{\textsuperscript{?}}
\newcommand{\ung}{*}

%%% monads and towers
\renewcommand{\l}{\lambda}
\newcommand{\mtype}[2]{\mathbb{#1}_{#2}}
\newcommand{\bind}{\star}
\newcommand{\unit}{\eta}
\newcommand{\nil}{\varepsilon}
\newcommand{\dnar}{\downarrow}
\newcommand{\upar}{\uparrow}
\newcommand{\reset}{\downupharpoons}
\newcommand{\hole}{[\,]}
\newcommand{\fsl}{\sslash}
\newcommand{\bsl}{\bbslash}
\newcommand{\msl}{\,\|\,}
\newcommand{\FSL}[1]{\ \stretchrel[600]{\ \fsl\ }{#1}}
\newcommand{\BSL}[1]{\ \stretchrel[600]{\ \bsl\ }{#1}}
\newcommand{\MSL}[1]{\ \stretchrel[600]{\ \msl\ }{#1}}

\makeatletter
\newcommand{\tower}[2][\my@btows]{%
  \ensuremath{\mathinner{%
      \begin{tabular}[c]{@{}>{$\displaystyle}c<{$}@{}}%
        #1 #2,\relax\noexpand\@eolst%
      \end{tabular}%
  }}%
}
\def\my@ttows #1,#2,#3\@eolst{%
  \ifx\relax#3\relax
    \strut\ensuremath{#1}\\
  \else
    \strut\ensuremath{#1}\ \hfil\vrule\hfil\ \ensuremath{#2}\\
    \hline
    \my@ttows #3\@eolst%
  \fi
}
\def\my@btows #1,#2\@eolst{%
  \ifx\relax#2\relax
    \ensuremath{#1}%
  \else
    \ensuremath{#1}\\\hline%
    \my@btows #2\@eolst%
  \fi
}
% tripartite tower of aribitrary depth
\newcommand{\ttower}[1]{\tower[\my@ttows]{#1}}
% bipartite tower of arbitrary depth
\newcommand{\btower}[1]{\tower{#1}}
\makeatother

\newcommand{\bitt}[2]{\btower{#1, #2}}
\newcommand{\bittt}[3]{\btower{#1, #2, #3}}
\newcommand{\bitttt}[4]{\btower{#1, #2, #3, #4}}

% tripartite 2-story tower
\newcommand{\tritt}[3]{%
  \ensuremath{\mathinner{%
      \begin{tabular}[c]{@{ }c@{ }}
        \strut\ensuremath{#1}\ \hfil\vrule\hfil\ \ensuremath{#2}\\
        \hline
        \strut\ensuremath{#3}\\
      \end{tabular}
  }}
}


%%%%% drawing %%%%%
\usepackage{tikz-qtree}

%%% tikz (for matrices)
\usepackage{tikz}
\usetikzlibrary{calc}
